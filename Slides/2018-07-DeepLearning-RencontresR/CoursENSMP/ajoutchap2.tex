\begin{eqnarray*}
\hat{c}(\x) & = & arg \max_k \pi (k | \x ),\\
            & = & arg \max_k p_k \cdot f_k(\x),\\
            & = & arg \min_k -2 \cdot \log p_k  + (\x - \bmu_k)^t \bSigma^{-1} (\x - \bmu_k).
\end{eqnarray*}

Remarquons que si les proportions $p_k$ sont toutes \'egales, alors cette r\`egle de
d\'ecision revient \`a affecter un vecteur forme \`a la classe la plus proche au
sens de la distance de \citeasnoun{Mahalonobis1936}
$$
\delta(\x,\bmu_k)=[(\x - \bmu_k)^t \bSigma^{-1} (\x - \bmu_k)]^{\frac{1}{2}}.
$$ 
Si, de plus, la matrice de variance est proportionnelle \`a la matrice identit\'e,
$$
\bSigma=\sigma \cdot I,
$$
alors la distance de Mahalanobis est \'equivalente \`a la distance euclidienne. Dans ce dernier
cas les classes sont suppos\'ees avoir une forme sph\'erique et un volume $\sigma$.
Lorsque les proportions sont diff\'erentes, le terme $-2 \cdot \log p_k$ biais la 
d\'ecision en faveur de la classes la plus probable {\em a priori}.


La r\`egle de d\'ecision peut s'exprimer sous une forme plus simple lorsque le terme
quadratique est d\'evelopp\'e, car $\x^t \bSigma^{-1} \x$ est une expression ind\'ependante
de l'indice de classe : 
\begin{eqnarray*}
\hat{c}(\x) & = & arg \min_k (\bSigma^{-1} \bmu_k)^t \cdot \x + (\frac{1}{2}{\bmu_k}^t\bSigma^{-1} \bmu_k + \log p_k),\\
            & = & arg \min_k \bw_k^t \cdot \x + w_{k0}.
\end{eqnarray*}

La fonction de d\'ecision est lin\'eaire et on parle d'analyse discriminante lin\'eaire, ce qui
implique, que les fronti\`eres s\'eparant deux  r\'egions voisines de d\'ecision, sont des hyperplans.
Consid\'erons  ${\cal R}_k$ et ${\cal R}_\ell$ deux r\'egions contigues :  la fronti\`ere entre ces
deux r\'egions est d\'ecrite par l'\'equation : 
$$
\bSigma^{-1} (\bmu_k-\bmu_\ell)\cdot (\x - \x_0)=0,
$$ 
o\`u 
$$
\x_0=\frac{1}{2}(\bmu_k - \bmu_ell) - \log \frac{p_k}{p_\ell} \frac{(\bmu_k-\bmu_\ell)}{(\bmu_k-\bmu_\ell)^t\bSigma^{-1} (\bmu_k-\bmu_\ell }.
$$
Ainsi la surface s\'eparatrice est un hyperplan orthogonal \`a $\bSigma^{-1} (\bmu_k-\bmu_\ell)$ et
passant par le point $\x_0$.

Notons que dans le cas particulier o\`u les proportions 
sont \'egales et la matrice de variance proportionnelle \`a la matrice identit\'e, alors
l'hyperplan est orthogonal \`a l'axes reliant les vecteur moyennes $\bmu_k$ et $\bmu_\ell$
et le point $\x_0$ est exactement au milieu du segment d\'efini par $\bmu_k$ et $\bmu_\ell$.
Si les proportions sont diff\'erentes cela revient \`a translater l'hyperplan vers la
classe la moins probable.

\begin{ex}\cite{Ripley1996}(suite de l'exemple \ref{ex:erreur})
Probabilit\'e d'erreur dans le cas de deux classes gaussiennes sous hypoth\`ese d'homosc\'edasticit\'e :
la r\`egle de d\'ecision prend la forme suivante
$$
\hat{c}(\x)=
\left \{ \begin{array}{l}
1 \ si \ A=(\bmu_1 - \bmu_2)^t \bSigma^{-1}(\x - \frac{1}{2}(\bmu_1 + \bmu_2)) > \log \frac{p_1}{p_2} , \\
2 \ sinon.\\
\end{array}
\right .
$$
Si $X$ appartient \`a la classe 1 alors on peut montrer que
$$
A \sim {\cal N}(\frac{1}{2}\delta^2,\delta^2)
$$
avec $\delta=[(\bmu_1 - \bmu_2)^t \bSigma^{-1} (\bmu_1 - \bmu_2)]^{\frac{1}{2}}$.
De m\^eme si $X$ appartient \`a la seconde classe alors 
$$
A \sim {\cal N}(-\frac{1}{2}\delta^2,\delta^2)
$$

Maintenant,  la probabilit\'e d'erreur peut s'\'ecrire comme : 
\begin{eqnarray*}
P(\mbox{Erreur})& = &  P(\X \in {\cal R}_2 | C=1) \cdot p_1 +  
			    P(\X \in {\cal R}_1 | C=2) \cdot p_2,\\
		& = &  p_1 \cdot P(A\leq \log \frac{p_1}{p_2}| C=1) + p_2 \cdot  P(A > \log \frac{p_1}{p_2}| C=2),\\
                & = &  p_1 \cdot \Phi(-\frac{1}{2} \delta + \frac{1}{\delta} \log \frac{p_1}{p_2}) + p_2 \cdot \Phi(-\frac{1}{2} \delta - \frac{1}{\delta} \log \frac{p_1}{p_2})
\end{eqnarray*}


\end{ex}

