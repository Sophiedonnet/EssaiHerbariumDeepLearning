\chapter{Classification automatique}

Une confusion r\'epandue existe entre les termes classement
et classification (respectivement ``classification'' et ``clustering''  en anglais).
Le classement pr\'esuppose l'existence de classes dont certains objets
sont connus, alors que la
classification tente de d\'ecouvrir une structure de classes 
qui soit ``naturelle'' aux donn\'ees. Dans la litt\'erature li\'ee \`a
la reconnaissance des formes, la distinction entre les deux approches
est souvent d\'esign\'ee par les termes
``apprentissage supervis\'e'' et ``non supervis\'e''. Une classification peut 
avoir diff\'erentes motivations   : compresser des informations,
d\'ecrire de mani\`ere simplifi\'ee de grandes masses de donn\'ees,
structurer un ensemble de connaissances,
r\'ev\'eler des structures, des causes cach\'ees,
r\'ealiser un diagnostic...
 
% baratin sur les fonctions discriminantes

Dans le contexte du classement, les fonctions discriminantes d\'efinissent
la similitude entre un vecteur forme et les diff\'erentes classes (chapitre 4).
Ces fonctions discriminantes peuvent \^etre construite \`a partir d'une
mesure de ressemblance entre les vecteurs forme. En classification le m\^eme
type d'approche  est \`a la base des m\'ethodes, et l'\'etape pr\'ealable
\`a toute classification consiste \`a d\'efinir une mesure
de  ressemblance entre les  objets (vecteurs formes). Traditionnellement
deux d\'emarches sont envisag\'ees :
\begin{itemize}
\item on peut dire que deux objets sont semblables s'ils  partagent une certaine
caract\'eristique. Consid\'erons le nombre de doigts d'un \^etre vivant et comparons
le singe et l'homme : sur ce crit\`ere de comparaison (et sur bien d'autres) les deux
esp\`eces seront jug\'ees semblables. Ce genre de d\'emarche aboutit \`a une classification
{\em monoth\'etique} base de l'approche aristot\'elicienne \cite{Sutcliffe1994}. Tous
les objets d'une m\^eme classe partagent alors un certain nombre de caract\'eristiques
(e.g. : ``Tous les hommes sont mortels'') ;   
\item on peut aussi mesurer la ressemblance en utilisant une mesure de proximit\'e (distance,
dissimilarit\'e). Dans ce cas la notion de ressemblance est mesur\'ee de fa\c{c}on plus
floue et deux objets d'une m\^eme classe poss\'ederont des caract\'eristiques ``proches''
au sens de la mesure utilis\'ee. Cette d\'emarche est dite {\em polyth\'etique}.
\end{itemize}

Le contexte de  ce chapitre est  l'approche polyth\'etique et
plus particuli\`erement,  les m\'ethodes de classification
qui mesurent la ressemblance \`a l'aide d'une distance..

Une classification am\`ene \`a r\'epartir l'ensemble des vecteurs forme en diff\'erentes
classes {\em homog\`enes}. La d\'efinition d'une classe et les relations entre classes 
peuvent \^etre tr\`es vari\'ees. Dans ce chapitre nous nous int\'eresserons aux deux 
principales structures de classification :
\begin{itemize}
\item
la partition, 
\item
la hi\'erarchie. 
\end{itemize}

\section{Partitions}


\begin{defi}
${\cal F}$ \'etant un ensemble fini, un ensemble $P=({\cal C}_1,{\cal C}_2,...,{\cal C}_K)$ de parties non vides 
de ${\cal F}$ est une partition si :
\begin{enumerate}
\item $\forall i\neq j$, ${\cal C}_i \cap {\cal C}_j=\emptyset,$
\item $\cup_i {\cal C}_i={\cal F}.$ 
\end{enumerate}
\end{defi}


Dans un ensemble  ${\cal F}=(\x_1,...,\x_N)$  partitionn\'e en $K$ classes,
chaque \'el\'ement de l'ensemble
appartient \`a une classe et une seule. Une mani\`ere pratique de d\'ecrire
cette partition $P$ consiste \`a utiliser une notation matricielle. Soit $\bc(P)$
la matrice caract\'eristique de la partition $P=({\cal C}_1,{\cal C}_2,...,{\cal C}_K)$ (ou matrice de classification) :

\[
\bc(P)=\bc=
\left(
\begin{array}{l l l} 
c_{11} & \cdots & c_{1K} \\
\vdots & \ddots & \vdots \\
c_{N1} & \cdots & c_{NK} \\
\end{array}
\right)
\] 
o\`u $c_{ik}=1$ si et seulement si $\x_i \in {\cal C}_k$, et $c_{ik}=0$ sinon. Remarquons que 
la somme de la $i$\ieme ligne est \'egale \`a 1 (un \'el\'ement appartient
\`a une seule classe) et la somme des valeurs de la $k$\ieme colonne vaut
$n_k$ le nombre d'\'el\'ements de la classe ${\cal C}_k$. On a donc 
$\sum_{k=1}^K n_k = N$.



La notion de partition dure repose sur une conception ensembliste
classique. Consid\'erant les travaux de \citeasnoun{Zadeh1965} sur les ensembles
flous, une d\'efinition du concept de partition floue semble
``naturelle''. La classification floue, d\'evelopp\'ee au d\'ebut des
ann\'ees 1970 \cite{Ruspini1969}, g\'en\'eralise une approche
classique en classification en \'elargissant la notion
d'appartenance \`a une classe.

Dans le cadre de la conception ensembliste classique, un individu $\x_i$
appartient appartient ou n'appartient pas \`a un ensemble donn\'e ${\cal C_k}$.
Dans la th\'eorie des sous-ensembles flous, un individu peut
appartenir \`a plusieurs classes avec diff\'erent degr\'es d'appartenance.
En classification cela revient autoriser les vecteurs formes
\`a appartenir \`a toutes les classes, ce qui se traduit par le
rel\^achement de la contrainte de binarit\'e sur les coefficients 
d'appartenance $c_{ik}$. Une partition floue est d\'efinie par
une matrice de classification floue $\bc=\{c_{ik}\}$ v\'erifiant les 
conditions suivantes :
\begin{enumerate}
\item $\forall k=1..K$, $\forall \x_i \in {\cal F}$, $c_{ik}\in [0,1]$.
\item $\forall k=1..K$, $0< \sum_{i=1}^N c_{ik} <N,$
\item $\forall \x_i \in {\cal F}$, $\sum_{k=1}^K c_{ik}.$ 
\end{enumerate}
La seconde condition traduit le fait qu'aucune classe
ne doit \^etre vide et la troisi\`eme exprime le 
concept d'appartenance totale.


\subsection{Crit\`eres et algorithmes}
%-------------------------------------------
Les concepts de partition et de classification polyth\'etique
\'etant pr\'ecis\'es, la question suivante \'emerge :
comment trouver une partition optimale d'un ensemble de donn\'ees,
lorsque la ressemblance entre deux individus est \'evalu\'ee par
une mesure de proximit\'e  ?

La premi\`ere chose \`a faire consiste \`a clarifier formellement
le sens du mot optimal. La solution g\'en\'eralement adopt\'ee est 
de choisir une mesure num\'erique de la qualit\'e d'une partition.
Cette mesure est parfois appel\'ee crit\`ere, fonctionnelle, ou
bien encore fonction d'\'energie. L'objectif d'une proc\'edure de classification
est donc de trouver la partition ou les partitions qui donnent la meilleure
 valeur (la plus petite ou la plus grande) pour un crit\`ere donn\'e.
 
Mais le nombre de partitions possibles, m\^eme pour un probl\`eme
de taille raisonnable, est \'enorme. En effet si l'on consid\`ere un
ensemble de $N$ objets \`a partitionner en $K$ classes, le nombre de
partitions possibles est :

\begin{equation}
NP(N,K)=\frac{1}{K  !}\sum_{k=0}^{K}(-1)^{k-1}  \cdot C_k^K \cdot k^N.
\end{equation}

\begin{ex}
Soit un ensemble de 8 objets que l'on d\'esire partager en 4 classes. Il existe
1701 partitions possibles   !
\end{ex}

Plut\^ot que de chercher la meilleure partition, celle qui donne 
la valeur optimale du crit\`ere, on utilise des m\'ethodes plus
rapides qui convergent vers des optima ``locaux'' du crit\`ere. Les 
partitions ainsi trouv\'ees sont souvent satisfaisantes.  


\subsection{Inertie intra-classe et partition}

De nombreux crit\`eres existent \cite{Gordon1980}. 
Certains peuvent \^etre li\'es, comme
nous le verrons dans la suite, au choix d'un mod\`ele pour 
l'ensemble des donn\'ees.  L'une des 
fonctions les plus utilis\'ee est la somme des variances intra-classes :

\begin{eqnarray*}
I_W & = & \sum_{k=1}^K \sum_{i=1}^N c_{ik} \| \x_i - \bm_k  \|^2 \\
    & = & trace(\bS_W)
\end{eqnarray*}
o\`u les $\bm_k$ sont les prototypes (centres) des classes et
les $c_{ik}$ sont les \'el\'ements d'une matrice de partition dure.
Le probl\`eme pos\'e est alors un probl\`eme d'optimisation sous
contraintes (li\'ees aux $c_{ik}$) :
\begin{equation}
(\hat{\bc}, \hat{\bm})=arg \min_{(\bc, \bm)}{ I_W(\bc,\bm)}
\end{equation}
o\`u $\bm$ repr\'esente l'ensemble des centres de gravit\'es.

\subsubsection{L'algorithme des centres mobiles}
\label{sec:algo}

Un algorithme tr\`es r\'epandu pour r\'esoudre ce probl\`eme est celui
des {\em k-means} ou centres mobiles. Historiquement, cet algorithme date
des ann\'ees soixante. Il a \'et\'e propos\'e par plusieurs chercheurs
dans diff\'erents domaines \`a des dates proches 
\cite{Edwards1965,Lloyd1957}. Cet algorithme bas\'e sur des consid\'erations
g\'eom\'etriques doit certainement son succ\`es \`a sa simplicit\'e et son efficacit\'e  :
\begin{enumerate}
\item Initialisation des centres : une m\'ethode r\'epandue consiste \`a initialiser les
centres avec les coordonn\'ees de $K$ points choisis au hasard.
\item Ensuite les it\'erations poss\`edent la forme altern\'ee suivante :
\begin{enumerate}
\item \'etant donn\'e $\bm_1,\cdots,\bm_K$, choisir les $c_{ik}$ qui 
minimisent $I_W$,
\item \'etant donn\'e $\bc=\{c_{ik}\}$, minimiser $I_W$ par rapport aux 
prototypes $\bm_1,\cdots,\bm_K$.
\end{enumerate}
\end{enumerate}

La premi\`ere \'etape affecte chaque $\x_i$ au prototype le plus proche, et
la seconde \'etape recalcule la position des prototypes en consid\'erant
que le prototype de la classe $i$ devient son vecteur moyenne. Il est possible de
montrer que chaque it\'eration fait d\'ecro\^{\i}tre le crit\`ere mais aucune 
garantie de convergence vers un maximum global n'existe en g\'en\'eral.
Si le crit\`ere des k-means est consid\'er\'e du point de vue de la recherche
d'une partition floue, c'est-\`a-dire si les contraintes sur les $c_{ik}$ sont
rel\^ach\'ees et deviennent $c_{ik}\in [0,1]$ \`a la place de $c_{ik}\in \{0,1\}$,
la partition optimale au sens du nouveau crit\`ere est celle qui est optimale 
pour le crit\`ere classique \cite{Selim1984}. En d'autre terme, il n'y a aucun 
int\'er\^et \`a consid\'erer des partitions floues, lorsqu'on travaille
avec le crit\`ere des k-means.

Cette forme d'algorithme altern\'e o\`u un certain crit\`ere est optimis\'e,
alternativement par rapport aux variables d'appartenance aux classes, puis
par rapport aux param\`etres d\'efinissant ces classes a \'et\'e intensivement
exploit\'e. Citons entre autre {\em les nu\'ees dynamiques} de Diday \cite{Diday1971}
et l'algorithme des {\em fuzzy c-means} \cite{Bezdeck1974}.

Notons que Webster \citeasnoun{Fisher1958} (\`a ne pas confondre avec Ronald Fisher)
avait propos\'e un algorithme trouvant la partition optimale,
au sens de la variance intra-classe, d'un ensemble de $N$ donn\'ees
unidimensionnelles en $O(N \cdot K^2)$ op\'erations en utilisant des m\'ethodes
issues de la programmation dynamique.  



\subsubsection{Une version adaptative des centres mobiles}

Une autre version des {\em k-means} \cite{Macqueen1967} consiste \`a modifier
les prototypes des classes en consid\'erant les donn\'ees une \`a une.
On parle alors d'algorithme adaptatif :
 

\begin{enumerate}
\item Les $K$ prototypes sont tir\'es au hasard parmi les $N$ points.
\item A l'it\'eration $q$, un individu $\x_i$ est choisi au hasard.
\begin{itemize}
\item D\'etermination du prototype le plus proche de $\x_i$ :
\[
\bm_k^q=\min_j{\|\x_i - \bm_j^q \|}.
\]
L'individu est affect\'e \`a la classe $k$.
\item Modification du prototype $\bm_k^q$ :
\[
\bm_k^{q+1}=\frac{\x_i + n_k^q \cdot \bm_j^q}{n_k^q +1},
\]
et
\[
n_k^{q+1}=n_k^q +1
\]
o\`u $n_k^q$ repr\'esente l'effectif de la classe $k$ \`a l'it\'eration $q$.
\end{itemize}
\end{enumerate}


Les algorithmes adaptatifs sont particuli\`erement ad\'equats lorsque
toutes les donn\'ees \`a classer ne sont pas disponibles \`a l'avance.
Les param\`etres d\'efinissant les classes peuvent alors \^etre
ajust\'es \`a l'apparition de chaque nouvelle donn\'ee sans trop de 
calculs. 

\subsubsection{Les nu\'ees dynamiques}

L'algorithme des nu\'ees dynamiques \cite{Diday1971} est une g\'en\'eralisation 
de centres mobiles. L'id\'ee de base consiste \`a remplacer les prototypes
$\bm_k$ (vecteurs de $\R^d$) par des \'el\'ements de  nature tr\`es diverse, 
nomm\'es noyaux. Le noyau $\bn_k$ d'une classe peut \^etre par exemple, un ensemble
de vecteurs forme de l'ensemble d'apprentissage, une droite, une loi de
probabilit\'e... La partition est obtenue en minimisant un crit\`ere
de la forme
\begin{equation}
W(\bn,\bc)=\sum_{k=1}^K \sum_{i=1}^N c_{ik} D(\x_i ,\bn_k )
\end{equation}
par une proc\'edure d'optimisation altern\'ee. Chaque it\'eration minimise
le crit\`ere en calculant successivement les noyaux en fixant la  partition
donn\'ee, puis la partition en fixant les noyaux.  
   

% ISODATA


\subsection{Mod\`eles de m\'elange et partitions}
%%%%%%%%%%%%%%%%%%%%%%%%%%%%%%%%%%%%%%%%%%%%%%%%%%%
De nombreux autres algorithmes et heuristiques qui optimisent d'autres
crit\`eres que celui de l'inertie intra-classes,  ou qui simplement 
produisent une partition, existent. L'approche probabiliste permet
de s'adapter \`a une grande vari\'et\'e de situations et
g\'en\'eralise certaines techniques usuelles 
(nu\'ees dynamiques, cartes de Kohonen). \\ 

L'approche probabiliste de la recherche de partitions fait
l'hypoth\`ese que l'ensemble des donn\'ees $\x=(\x_1,...,\x_N)$ est la
r\'ealisation d'un \'echantillon de $N$ variables al\'eatoires ind\'ependantes
de m\^eme loi $f$, prenant leurs valeurs dans $\R^d$. La connaissance 
de cette loi $f$ doit permettre de s\'eparer ``naturellement'' les $N$
observations en $k$ classes.  

Les m\'elanges finis de densit\'e sont les distributions de probabilit\'e
les plus  utilis\'ees dans ce contexte :
\begin{equation}
f(\x)=\sum_{k=1}^K p_k f_k(\x | \theta_k),
\end{equation}
avec $\sum_{k=1}^K p_k=1$. Ce genre de densit\'e appara\^{\i}t naturellement
lorsque la population consid\'er\'ee est form\'ee de plusieurs
sous-populations qui ont des densit\'es diff\'erentes. 
Ceci explique l'int\'er\^et de ce mod\`ele en classification. 
Les mod\`eles de m\'elange,  de loi de Gauss ou de Bernoulli, sont
les mod\`eles le plus souvent utilis\'es  dans le contexte de la
classification automatique.

Dans ce contexte, deux approches ont \'et\'e propos\'ees :
\begin{itemize}
\item
l'approche m\'elange,
\item
l'approche classification.
\end{itemize} 
La premi\`ere approche consid\`ere effectivement que les individus
observ\'es sont les r\'ealisation d'un m\'elange de densit\'e, 
alors que la seconde approche traite chacune des sous-population
de mani\`ere s\'epar\'ee. 

\subsubsection{Approche  m\'elange}

Si l'on consid\`ere que les vecteur forme observ\'es 
$(\x_1,\cdots,\x_N)$ sont des r\'ealisations ind\'ependantes
d'une loi m\'elange :
$$
f(\x)=\sum_{k=1}^K p_k f_k(\x | \theta_k),
$$
alors la log-vraisemblance des param\`etres 
$\Phi=(p_1,\cdots,p_{K-1},\theta_1,\cdots,\theta_K)$
s'exprime comme :
$$
L(\Phi ; \x_1,\cdots,\x_N)= \sum_{i=1}^{N} \log{ \left(p_k f_k(x_i | \theta_k)    \right )}.
$$
En g\'en\'eral l'algorithme EM (chapitre 3) est utilis\'e pour trouver
des estimateurs du m.v. (maximum de vraisemblance). Dans le cadre de cette
approche, une partition des donn\'ees peut \^etre obtenue \`a partir 
des estimateur du m.v. en affectant chaque vecteur forme \`a la 
composante du m\'elange (donc la classe) la plus probable. La 
probabilit\'e conditionnelle que $\x_i$ soit issu de la 
$k$\ieme composante est donn\'ee par :
\begin{equation}
\label{eq:tkxi}
t_k(\x_i)=\frac{\hat{p}_k f_k(x_i | \hat{\theta}_k) }{\sum_{\ell=1}^{K}
\hat{p}_{\ell} f_{\ell}(x_i | \hat{\theta}_{\ell})}.
\end{equation}


\subsubsection{Approche classification}

Une  autre approche possible, dans une optique de partitionnement
de l'\'echantillon, consiste \`a consid\'erer 
directement la partition comme le param\`etre inconnu. 
Les pionniers de cette approche sont \citeasnoun{Scott1971}
et \citeasnoun{Schroeder1976}. Dans ce contexte le probl\`eme \`a r\'esoudre
peut \^etre formul\'e comme suit : \'etant donn\'e un \'echantillon
de taille $N$,  $(\x_1,...,\x_N)$, rechercher une partition dure
$P=({\cal C}_1,\cdots,{\cal C}_K)$, K \'etant suppos\'e connu, telle que 
chaque classe ${\cal C}_k$ soit assimilable \`a un sous-\'echantillon suivant
la loi $f_k(.|\theta_k)$.


Le crit\`ere consid\'er\'e alors n'est plus la vraisemblance de 
l'\'echantillon, mais la vraisemblance classifiante, soit le 
produit des vraisemblance sur les classes. La log-vraisemblance
s'\'ecrit alors :
\begin{equation}
CML(\Phi ,\bc ; \x_1,\cdots,\x_N)=\sum_{k=1}^K \sum_{i=1}^n c_{ik }\log{ \{f_k(\x_i | \theta_k  \} }
\end{equation}
avec $\Phi=(\theta_1,\cdots,\theta_K)$ et $\bc=\{ c_{ik}\}$ une matrice 
de partition dure qui d\'efinit $K$ classes (ou sous \'echantillons). 
Ce crit\`ere est la log-vraisemblance associ\'ee \`a $K$ \'echantillons
s\'epar\'es de taille fix\'ee.


La vraisemblance classifiante ne fait pas appara\^{\i}tre explicitement
la notion de proportions entre les diff\'erentes sous-populations
et tend en pratique \`a produire des
partitions o\`u les classes sont de tailles comparables. En fait
le crit\`ere suppose implicitement que toutes les sous-populations
sont de m\^eme taille. Cette limitation
a incit\'e \citeasnoun{Symons1981} \`a p\'enaliser la vraisemblance classifiante
par un terme prenant en compte les proportions $(p_1,\cdots,p_K)$ des
diff\'erents sous-\'echantillons :
\begin{equation}
CML'(\Phi',\bc)=CML(\Phi, \bc)+ \sum_{k=1}^K n_k \log{p_k}
\end{equation}
o\`u $\Phi'=(\Phi,p_1,\cdots,p_K)$ et $n_k$ est l'effectif de la $k$\ieme
classe.
Notons qu'en introduisant les variables $c_{ik}$ dans le terme de p\'enalit\'e,
la vraisemblance classifiante p\'enalis\'ee s'\'ecrit
\begin{eqnarray*}
CML'(\Phi',\bc) & = & CML(\Phi, \bc ) + \sum_{k=1}^K \sum_{i=1}^N c_{ik} \log{p_k}\\
             & = &\sum_{k=1}^K \sum_{i=1}^N c_{ik} \log{\{ p_k f_k( \x_i | \theta_k ) \} },\\
             & = &\sum_{k=1}^K \sum_{i=1}^{n_k} \log  P(\x_{i,k}, k; \Phi'),
\end{eqnarray*}
o\`u $\x_{i,k}$ d\'enote un vecteur forme de la classe $k$. Ce dernier crit\`ere
s'interpr\`ete comme la log-vraisemblance d'un \'echantillon al\'eatoire
$\{(\x_1,c(\x_1)),\cdots,(\x_N,c(\x_N))\}$ o\`u  \`a la diff\'erence d'un probl\`eme de
classement les \'etiquettes $c(\x_i)$ n'ont pas \'et\'e observ\'ees. 

Ces deux crit\`eres peuvent \^etre maximis\'es par une version classificatoire
de l'algorithme EM : {\em Classification EM algorithm}. L'algorithme CEM
a \'et\'e propos\'e par \citeasnoun{Celeux1992a}. 

Une it\'eration cet algorithme se d\'ecompose ainsi :
\begin{itemize}
\item {\bf \'Etape E (estimation) :} Calcul des probabilit\'es $t_k(\x_i)^q$
(cf. \'equation \ref{eq:tkxi}) pour
chaque $\x_i$.

\item {\bf \'Etape C (classification) :} Chaque $\x_i$ est affect\'e \`a la 
composante du m\'elange de plus forte probabilit\'e a posteriori. Une partition
$P^{q+1}$ est donc d\'efinie caract\'eris\'ee par la matrice $\bc=\{ c_{ik} \}$ avec
$c_{ik}=1$ si $k=arg \max_\ell{ t_\ell(x_i)^q}$ et $c_{ik}=0$ sinon.

\item {\bf \'Etape M (maximisation) :} Calcul des estimateurs du m.v. de 
$\Phi^{q+1}$ sur la base des sous-\'echantillons pr\'ecis\'es par la matrice
de classification dure $\bc$.
\end{itemize}

L'algorithme CEM g\'en\`ere une suite $CML'(\Phi^q,\bc^q)$ croissante qui
atteint son maximum en un nombre fini d'it\'erations \cite{Celeux1992a}.


%Lien avec les centres mobiles.


L'algorithme CEM est un algorithme tr\`es g\'en\'eral de classification
qui permet d'optimiser de nombreux crit\`eres de classification de 
type inertiels suivant les mod\`eles gaussiens consid\'er\'es.
Prenons par exemple le mod\`ele 
gaussien le moins contraint, pour lequel les classes sont de tailles 
diff\'erentes et poss\`edent une
matrice de variance covariance quelconque,
l'algorithme CEM maximise alors le crit\`ere
de vraisemblance classifiante p\'enalis\'ee. Si toutes les proportions
sont fix\'ees \'egales, le crit\`ere optimis\'e est alors simplement
la vraisemblance classifiante. Un autre cas particulier int\'eressant
est celui o\`u les densit\'es $f_k(.|\theta_k)$ du m\'elange sont des
gaussiennes de vecteur moyenne $\bmu_k$ et de matrice de variance covariance
$\bSigma_k=\lambda \cdot I$, m\'elang\'es en proportions \'egales. En 
effet le crit\`ere optimis\'e est la somme des variances intra-classes,
\begin{eqnarray*}
 CML(\Phi,\bc ) & = &  \sum_{k=1}^K \sum_{i=1}^N c_{ik}
                 \log{  (2\pi \det{|\bSigma_k|})^{-\frac{d}{2}}
                 \exp{(-\frac{1}{2}(\x-\bmu_k)^t{(\lambda \cdot I)}^{-1} (\x-\bmu_k))}}\\
             & = &  -\frac{1}{2 \lambda}\sum_{k=1}^K \sum_{i=1}^N c_{ik}
                 (\x-\bmu_k)^t (\x-\bmu_k) + Cst,\\
            & = & -\frac{1}{2 \lambda} trace(\bS_W) + Cst, 
\end{eqnarray*}
et l'algorithme CEM,  avec ce mod\`ele, est exactement l'algorithme des
centres mobiles pr\'esent\'e dans la section \ref{sec:algo}. 

Diff\'erentes \'etudes \cite{Celeux1992b} ont montr\'e que l'approche classification
introduisait un biais dans l'estimation des param\`etres. En effet, cette approche
estime les param\`etres du m\'elange sur la base des classes, alors que les classes
sont disjointes et constituent en fait des \'echantillons tronqu\'es des
composantes du m\'elange. Ce ph\'enom\`ene a tendance \`a surestimer les diff\'erences
entre les moyennes, et \`a sous estimer les variances et les diff\'erences entre 
proportions. Ces inconv\'enients ne sont pas r\'edhibitoires si les classes
sont bien s\'epar\'ees et les proportions du m\^eme ordre de grandeur. 




\subsubsection{Liens avec la classification floue}
%+++++++++++++++++++++++++++++++++++++++++++++++++++++++

Dans le cadre de la reconnaissance des formes, l'algorithme EM
pour les mod\`eles de m\'elanges peut \^etre interpr\'et\'e comme un
algorithme d'optimisation altern\'ee d'un certain crit\`ere 
\cite{Hathaway1986,Celeux1994}.

Si les probabilit\'es a posteriori $t_k(\x_i)$ sont consid\'er\'ees comme 
des variables not\'ees $c_{ik}$, la log-vraisemblance $L(\Phi;\x)$ 
devient une fonction du vecteur $\Phi$ et des $c_{ik}$ que nous noterons :
\begin{equation}
L(\bc,\Phi)=\sum_{i=1}^N \sum_{k=1}^K c_{ik} \log{p_k f_k(\x_i | \theta_k)}
           - \sum_{i=1}^N \sum_{k=1}^K c_{ik} \log{ c_{ik} },
\end{equation} 
avec $\bc=\{c_{ik} :0 \leq c_{ik} \leq 1, \sum_{k=1}^K c_{ik}=1, \sum_{i=1}^N
c_{ik}>0 (1\leq i \leq N, 1 \leq k \leq K)   \}$.


Consid\'erons le probl\`eme qui consiste \`a maximiser $L(\bc,\Phi)$ par
rapport aux variables $\Phi$ et $\bc$. Il s'agit d'un probl\`eme
classique d'optimisation sous contraintes. Une m\'ethode d'optimisation
possible consiste \`a s\'eparer les variables en deux groupes et \`a optimiser
le crit\`ere alternativement par rapport \`a un groupe en gardant fixe  les 
valeurs des variable de l'autre groupe. Dans le cas du crit\`ere $L(\bc,\Phi)$,
pour la $q$\ieme it\'eration il est possible d'optimiser  alternativement 
par rapport \`a $\bc$ puis \`a $\Phi$ :
\begin{enumerate}
\item Maximisons $L(\bc,\Phi)$ par rapport \`a $\bc$ : le lagrangien s'\'ecrit
\begin{equation}
{\cal{L}}(\bc)=L(\bc,\Phi) + \sum_{i=1}^N \lambda_i (\sum_{k=1}^K (c_{ik}-1)),
\end{equation}
o\`u les $\lambda_i$ sont les coefficients de Lagrange correspondant aux 
contraintes 
$$\sum_{k=1}^K c_{ik}=1.$$ 
Les conditions n\'ecessaires d'optimalit\'e am\`enent les \'equations suivantes :
\[
\left \{ \begin{array}{l}
	 				\frac{\partial{\cal{L}}}{\partial c_{ik}}=
\log{(p_kf_k(\x_i|\theta_k))} -1 -\log{c_{ik}} + \lambda_i = 0 ; \\
				  \sum_{k=1}^K c_{ik}=1 ;
	               \end{array}
     		\right .
\]

ce qui donne,

\[
\left \{ \begin{array}{l}
	 				c_{ik}=
\exp{\{ \log(p_kf_k(\x_i|\theta_k)) -1 + \lambda_i \}};  \\
				  \sum_{k=1}^K \exp{\{ \log(p_kf_k(\x_i|\theta_k)) -1 + \lambda_i \}} =1;
	          \end{array}
     		\right .
\]




Ainsi les nouvelles valeurs des $c_{ik}$ sont :
\begin{equation}
c_{ik}=\frac{p_k f_k(\x_i | \theta_k)}{f(\x_i)}.
\end{equation}


\item La maximisation de  $L(\bc,\Phi)$ par rapport \`a $\Phi$ est 
\'equivalente l'\'etape M de l'algorithme EM.  
\end{enumerate}

Ces deux \'etapes qui visent \`a maximiser le crit\`ere $L(\bc,\Phi)$ sont
identiques aux deux \'etapes de l'algorithme EM appliqu\'e \`a un m\'elange
de distribution de probabilit\'e.

Si l'on consid\`ere $\bc$ comme une matrice de classification floue (elle en
a toutes les caract\'eristiques), l'algorithme EM peut \^etre interpr\'et\'e
comme un algorithme de classification floue. 


Remarquons que le crit\`ere optimis\'e s'\'ecrit comme la somme de 
deux termes :
\begin{itemize}
\item Dans la terminologie utilis\'ee en classification automatique,
le premier est appel\'e ``vraisemblance classifiante floue'' (avec proportions
libres). Plusieurs algorithmes de 
classification automatique existent qui visent \`a trouver la partition
dure qui optimise la vraisemblance classifiante. 
\item Le second terme 
peut \^etre consid\'er\'e comme une entropie, ou encore une mesure 
de floue de la partition. Ce second terme est maximum si la partition
obtenue est compl\`etement floue et minimum (nul en l'occurence) 
$\bc$ est une matrice de partition dure. 
\end{itemize}
Au vu des remarques pr\'ec\'edentes, l'algorithme EM peut \^etre consid\'er\'e
comme un algorithme de classification flou qui optimise un
crit\`ere de classification p\'enalis\'e par une entropie. 

\section{Hi\'erarchies}

\begin{defi}
${\cal F}$ \'etant un ensemble fini,
un ensemble $H$ de parties non vides de ${\cal F}$ est une hi\'erarchie si :
\begin{enumerate}
\item ${\cal F} \in H$
\item $\forall x \in {\cal F}$, $\{ x \} \in H$ 
\item $\forall h,h' \in H$,
$h \cap h'=\emptyset$ ou $h\subset h'$ ou $h'\subset h$
\end{enumerate}
\end{defi}
Une hi\'erarchie peut \^etre vue comme un ensemble de partitions 
embo\^\i t\'ees. Graphiquement une hi\'erarchie est souvent repr\'esent\'ee
par une structure arborescente appel\'ee  dendogramme. 

\begin{ex}Exemple de hi\'erarchie :
en biologie les diff\'erente races d'animaux sont regroup\'ees en esp\`eces, qui sont elles
m\^eme regroup\'ees en grande famille...  
\end{ex}

Une hi\'erarchie peut \^etre obtenue par deux types de m\'ethodes, selon que 
l'arbre est construit en commen\c cant par les feuilles ou bien la racine :
\begin{itemize}
\item la classification ascendante (``agglom\'erative'') consid\`ere
initialement chaque vecteur forme comme une classe. \`A chaque
it\'eration les deux classes les plus proches sont agr\'eg\'ees
pour former une nouvelle classe. Le processus se termine naturellement
lorsqu'il ne reste qu'une seule classe.
\item la classification descendante (``divisive'' en anglais), 
part d'une seule classe (l'ensemble des vecteurs forme)
partage celle-ci en deux. L'op\'eration est r\'ep\'et\'ee
\`a chaque it\'eration jusqu`a ce toutes les classes obtenues
contiennent un unique vecteur forme. 
\end{itemize}             

Il existe un parall\`ele int\'eressant entre la notion de distance ultram\'etrique et la notion
de hi\'erarchie. Une distance utlram\'etrique $\delta$ v\'erifie toutes les propri\'et\'es qui 
d\'efinissent une distance classique et satisfait en plus l'in\'egalit\'e
$$
\delta(\x,\z) \leq \max \left ( \delta(\x,\z),\delta(\z,\y) \right ),
$$
plus forte que l'in\'egalit\'e triangulaire. Lorsqu'on dispose d'une hi\'erarchie, on peut
interpr\'eter le nombre minimum d'emboitements n\'ecessaires pour que deux vecteurs forme
appartiennent \`a une m\^eme classe, comme une dissimilarit\'e. Il est alors possible de 
montrer que cette dissimilarit\'e est une distance ultram\'etrique.
Ainsi, il est possible d'intepr\'eter le probl\`eme
de la classification hi\'erarchique comme la recherche d'une ultram\'etrique $\delta$  proche
de $d$, la dissimilarit\'e  utilis\'ee sur ${\cal F}$ l'ensemble \`a classer.    

\subsection{Classification ascendante hi\'erarchique}
%%%%%%%%%%%%%%%%%%%%%%%%%%%%%%%%%%%%%%%%%%%%%%%%%%%%
Le principe des algorithmes de classification hi\'erarchique
ascendante est tr\`es simple :
\begin{description}
\item[Initialisation :] chaque \'el\'ement de ${\cal F}$ constitue
une classe. Une ``distance'' $D$ est calcul\'ee entre toutes les
classes. 
\item[Tant que] nombre de classes $>$ 1
\begin{itemize}
\item 
regrouper les deux classes les plus proches au sens de la ``distance''
$D$,
\item
calcul des ``distances'' entre la nouvelle classe et les autres.
\end{itemize} 
\end{description}

La ``distance'' $D$ entre deux partie $h$ et $h'$ de ${\cal F}$, peut
\^etre d\'efinie de nombreuses mani\`eres \`a partir d'une
mesure de dissimilarit'e $d$ sur ${\cal F}$.
 

\subsubsection{Crit\`eres d'agr\'egation}
%------------------------------------------------------

La ``distance'' $D$ est couramment appel\'ee crit\`ere d'aggr\'egation.
Quatre variantes sont principalement utilis\'ees :
\begin{itemize}
\item
le crit\`ere du lien minimum (``single link'')  :
$$
D(h,h')= \min \left [     d(\x , \y) / \x \in h \ et \ \y \in h' \right ],
$$
\item
le crit\`ere du lien maximum (``complete link'')  :
$$
D(h,h')= \max \left [     d(\x , \y) / \x \in h \ et \ \y \in h' \right ],
$$
\item
le crit\`ere de la distance moyenne (``group average'')  :
$$
D(h,h')= \frac{\sum_{i=1}^{n_{h}}  \sum_{j=1}^{n_{h'}}  d(\x_i , \x_j)}{n_{h} \cdot n_{h'}}, 
$$
\item
le crit\`ere de \citeasnoun{Ward1963}
$$
D(h,h')= \frac{n_{h} \cdot n_{h'}}{n_{h} + n_{h'}}\|\bm_{h} - \bm_{h'} \|^2.
$$
\end{itemize}

\subsubsection{Crit\`ere d'inertie intra classe et m\'ethode de Ward}
%------------------------------------------------------

Lorsque l'on dispose d'une partition en $K$ classe, le crit\`ere
d'inertie intra-classe mesure son homog\'en\'eit\'e :
\begin{eqnarray*}
I_W  & = & trace(\bS_W),\\
     & = & \sum_{k=1}^K \sum_{i=1}^{n_k} (\x_{ik} - \bm_k )^t (\x_{ik} - \bm_k).
\end{eqnarray*} 

Consid\'erons deux partitions :
\begin{itemize}
\item $P=\left ( {\cal C}_1, \cdots , {\cal C}_K  \right)$,

\item et $P'$, la partition obtenue en fusionnant les classes 
${\cal C}_k$ et ${\cal C}_{\ell}$. 
\end{itemize}

On peut montrer que la diff\'erence entre l'inertie des deux partitions
est \'egale au crit\`ere d'aggr\'egation de Ward :
$$
I_{W'}-I_W = \frac{n_{k} \cdot n_{\ell}}{n_{k} + n_{\ell}}\|\bm_{k} - \bm_{\ell} \|^2.
$$
Ainsi,\`a chaque \'etape de l'algorithme de Ward choisit une
nouvelle partition qui limite l'augmentation de l'inertie intra-classe.
Notons que cette propri\'et\'e ne garantie pas
l'optimisation globale du crit\`ere.
\subsection{Classification descendante hi\'erarchique}
%%%%%%%%%%%%%%%%%%%%%%%%%%%%%%%%%%%%%%%%%%%%%%%%%%%%%%%

La classification descendante est beaucoup moins populaire que les
m\'ethodes d\'ecrites pr\'ec\'edemment. En th\'eorie, la premi\`ere
\'etape d'une m\'ethode descendante doit comparer les $2^{N-1}-1$ partitions
possibles des $N$ vecteurs forme, en deux classes. Pour \'eviter des 
calculs impossibles, une solution consiste \`a appliquer une m\'ethode 
de partitionnement pour obtenir les deux classes. En r\'ep\'etant
ce processus r\'ecursivement sur chaque classe obtenue, il en
r\'esulte une hi\'erarchie. 

  









