\nocite{Gallinari1988}

\chapter{Extraction et s\'election de caract\'eristiques}
En reconnaissance des formes, les algorithmes de classement 
classent des vecteurs forme $\x_i$ qui sont des descriptions 
d'objets. Cette approche fait l'hypoth\`ese qu'une 
description coh\'erente de l'ensemble des objets existe. Dans ce contexte, la
question suivante \'emerge : comment choisir les caract\'eristiques
descriptives. Ce chapitre envisage quelques r\'eponses \`a cette
interrogation.

Dans un premier temps, pr\'ecisons le probl\`eme. Si l'on suppose 
que chaque objet est caract\'eris\'e par $d$ variables
quantitatives (poids, taille, vitesse...), et appartient
\`a une classe parmi $K$,  le probl\`eme pr\'ec\'edent peut 
alors se reformuler comme suit :
\begin{itemize}
\item quelles variables parmi les $d$ disponibles sont les 
plus discriminantes ?
\item comment extraire de nouvelles variables, \`a partir des
variables initiales, qui soient discriminantes ? 
\end{itemize}
Utilis\'ees comme pr\'etraitement des donn\'ees, ces deux approches, 
extraction  et s\'election de caract\'eristiques,
permettent de faciliter la t\^ache du classifieur. La r\'eduction
de dimension (passage de $d$ \`a $q$ variables avec $q<d$) est
par exemple d'une grande aide pour les m\'ethodes non param\'etriques
de classement bas\'ees sur les estimateurs \`a noyaux (chapitre 3). En
effet, ces derni\`eres m\'ethodes sont inefficaces dans les espaces
de grandes dimensions et gagnent \`a \^etre utilis\'ees apr\`es 
r\'eduction de dimension. 

Dans le cadre d'une analyse exploratoire  des donn\'ees, les techniques
d'extraction et de s\'election de caract\'eristiques permettent aussi
de visualiser les vecteurs forme. 

Dans ce chapitre nous pr\'esentons :
\begin{itemize}
\item l'analyse factorielle discriminante, qui est une technique
d'extraction de caract\'eristiques cherchant des combinaisons lin\'eaires
des variables initiales ;
\item quelques m\'ethodes de positionnement multidimensionel (``multidimensional scaling''),
qui ne sont pas sp\'ecifique au probl\`eme du classement, mais permettent
la visualisation des vecteurs forme ;
\item les principales  m\'ethodes de s\'election de caract\'eristiques.
\end{itemize}
 


\section{Analyse factorielle discriminante}
% historique et but de la m\'ethode\
L'analyse factorielle discriminante vise \`a trouver des nouvelles
variables, appel\'ees facteurs, combinaisons lin\'eaires des 
variables initiales, 
qui permettent de distinguer le mieux possible, les 
$K$ groupes de vecteurs forme. Historiquement la m\'ethode
a \'et\'e introduite par \citeasnoun{Fisher1936} dans le cas de deux
classes. La g\'en\'eralisation au cas multi-classes est due 
\`a \citeasnoun{Rao1948}

Pr\'ecisons quelques notations utiles \`a  la pr\'esentation
de la m\'ethode :
\begin{itemize}
\item
les vecteurs moyennes des $K$ groupes sont not\'es
$$
\bm_k=\frac{1}{n_k}\sum_{i=1}^{n_k} \x_{i,k}.
$$
\item
$\bm$ d\'enote le vecteur moyenne de l'ensemble des vecteurs formes, 
$$
\bm= \frac{1}{N}\sum_{k=1}^{n_k} n_k \cdot \bm_k,
$$
\item
la matrice d'inertie intra-classe $\bS_W$ ($W$ comme ``within'') 
est d\'efinie par :
\begin{eqnarray*}
\bS_W &=&  \sum_{k=1}^K \sum_{i=1}^{n_k} (\x_{i,k}-\bm_k)(\x_{i,k}-\bm_k)^t,
\end{eqnarray*}
\item
la matrice d'inertie inter-classe $\bS_B$ ($B$ comme ``between'') est 
d\'efinie par :
\begin{eqnarray*}
\bS_B & = & \sum_{k=1}^K n_k \cdot (\bm_k - \bm)(\bm_k - \bm)^t ,
\end{eqnarray*}
\item
et enfin la matrice d'inertie totale $\bS_T$ s'\'ecrit :
\begin{eqnarray*}
\bS_T &=& \sum_{k=1}^K \sum_{i=1}^{n_k} (\x_{i,k}-\bm) (\x_{i,k}-\bm)^t \\
    &=& \sum_{k=1}^K \sum_{i=1}^{n_k} (\x_{i,k}-\bm_k+\bm_k-\bm)(\x_{i,k}-\bm_k+\bm_k-\bm)^t\\
    &=&     \sum_{k=1}^K \sum_{i=1}^{n_k} (\x_{i,k}-\bm_k)(\x_{i,k}-\bm_k)^t +
           \sum_{k=1}^K n_k \cdot (\bm_k - \bm)(\bm_k - \bm)^t\\
   &=& \bS_W + \bS_B.
\end{eqnarray*}
Notons que cette derni\`ere \'egalit\'e est une application du th\'eor\`eme de
Huygens et  une g\'en\'eralisation de la formule de l'analyse de la variance.
\end{itemize}


\subsection{Discriminant lin\'eaire de Fisher}
Supposons un ensemble d'apprentissage 
${\cal F}$  compos\'e de $n_1$ vecteurs forme $\x_{i,1}$ de la classe
${\cal C}_1$ et $n_2$ vecteurs forme $\x_{j,2}$ de la classe
${\cal C}_2$. Une combinaison lin\'eaire $y$ d'un vecteur $\x$
$$
y=\bw^t \cdot \x,
$$
est un scalaire. Dans le but de poser un probl\`eme poss\'edant une
solution unique, on impose $\| \bw \|=1$.


Le but de l'analyse discriminante de Fisher
est de trouver un facteur $\bw$ tel que les
sur laquelle les
projections des vecteurs $\x_{i,1}$ soit bien s\'epar\'ees 
des projections des vecteurs $\x_{j,2}$.

Notons :
\begin{itemize}
\item 
$y_{i,k}$ les projections des $\x_{i,k}$ : $y_{i,k}= \bw^t \cdot \x_{i,k}$,
\item
$\tilde{m}_k$ les projections des vecteurs moyennes :  $\tilde{m}_k= \bw^t \cdot \bm_{k}$,
\item et $\tilde{s}_k^2$ l'inertie des projections des vecteurs formes
du groupe $k$ :
$$
\tilde{s}_k^2= \sum_{i=1}^{n_k} (y_{i,k}-\tilde{m}_k)^2.
$$
\end{itemize}


Le crit\`ere mesurant la qualit\'e de cette s\'eparation sur
l'axe $\bw$, propos\'e par Fisher, est le rapport de
l'inertie inter-classes sur l'inertie intra-classes 
des projections des vecteurs forme de ${\cal F}$ :
$$
J(\bw)=\frac{(\tilde{m}_1 -\tilde{m}_2)^2}{\tilde{s}_1^2+\tilde{s}_2^2}.
$$
Intuitivement les deux groupe projet\'es seront bien s\'epar\'es,
si l'\'ecart entre $\tilde{m}_1$ et  $\tilde{m}_2$ est grand, c'est-\`a-dire
si les centre de gravit\'e des groupes sont les plus \'eloign\'es possibles
en projection, et si chaque classe projet\'ee forme le groupe le plus
compact possible autour de leur centre de gravit\'e respectif
(inertie intra-classe petite). En d'autres
termes, la s\'eparation sera d'autant plus nette que le crit\`ere
$J(\bw)$ sera important. 

Dans le contexte des tests d'hypoth\`eses, lorsque l'on d\'esire
tester l'\'egalit\'e des moyennes  $\tilde{m}_1$ et  $\tilde{m}_2$ 
de deux groupes, l'analyse
de la variance am\`ene \`a d\'efinir la r\'egion critique suivante :
$$
J=\frac{(\tilde{m}_1- \tilde{m}_2)^2}{\tilde{s}_1^2+\tilde{s}_2^2} > A.
$$
On peut montrer que $J$ est proportionnelle \`a une variable al\'eatoire
qui suit une loi de Fisher Snedecor ${F}_{K-1,N-K}$. La maximisation
du crit\`ere $J(\bw)$ revient donc \`a trouver un sous espace de projection
o\`u le test d'\'egalit\'e des moyennes sera le plus pessimiste possible.


La recherche du vecteur $\bw$ peut donc \^etre formul\'e comme un
probl\`eme d'optimisation sous contrainte :
$$
\left \{ \begin{array}{l}
\bw = arg \max_{\boldv}  J(\boldv)  \\
\| \bw \|=1    \\
\end{array}
\right.
$$ 
En faisant appara\^{\i}tre dans le crit\`ere $J(\bw)$ les vecteurs forme $\x_{i,k}$,
il vient
$$
\left \{ \begin{array}{l}
\bw = arg \max_{\boldv} \frac{\boldv^t(\bm_1-\bm_2)(\bm_1-\bm_2)^t \boldv}{\boldv^t \bS_W \boldv}  \\
\| \bw \|=1    \\
\end{array}
\right.
$$ 
Remarquons que 
\begin{equation}
\label{eq:SBFisher}
\bS_B= \frac{n_1 n_2}{N^2} (\bm_1-\bm_2)(\bm_1-\bm_2)^t.
\end{equation}
En effet si l'on consid\`ere que les vecteurs forme sont centr\'es ($\bm=0$),
on a 
$$
\bm=\frac{n_1}{N}\bm_1 +  \frac{n_2}{N}\bm_2=0.
$$
On peut donc \'ecrire que 
$$
\left \{ \begin{array}{l}
\bS_B = \frac{n_1}{N} \bm_1 \bm_1^t +  \frac{n_2}{N} \bm_2 \bm_2^t - 
       \underbrace{\left ( \frac{n_1}{N} \bm_1 \bm_1^t +  \frac{n_2}{N} \bm_2 \bm_1^t    \right )}_0 = -\frac{n_2}{N}\bm_2 (\bm_1 - \bm_2)^t \\
\\
\bS_B = \frac{n_1}{N} \bm_1 \bm_1^t +  \frac{n_2}{N} \bm_2 \bm_2^t - 
       \underbrace{\left ( \frac{n_1}{N} \bm_1 \bm_2^t +  \frac{n_2}{N} \bm_2 \bm_2^t    \right )}_0 = \frac{n_1}{N}\bm_1 (\bm_1 - \bm_2)^t  \\ 
\end{array}
\right.
$$
En moyennant ($n_1$ fois la premi\`ere expression plus $n_2$ fois la deuxi\`eme) on montre l'\'equation \ref{eq:SBFisher} qui nous permet de poser
le probl\`eme de la recherche de $\bw$ sous la forme : 
$$
\left \{ \begin{array}{l}
\bw = arg \max_{\boldv} \frac{\boldv^t \bS_B  \boldv}{\boldv^t \bS_W \boldv}  \\
\| \bw \|=1    \\
\end{array}
\right.
$$ 
On voit dans que $\lambda_{max}$ la valeur maximum du crit\`ere v\'erifie :
\begin{eqnarray*}
\lambda_{max} \cdot \bw^t \bS_W \bw & = &  \bw^t \bS_B \bw \\
\lambda_{max} \cdot \bS_W \bw& = &  \bS_B \bw
\end{eqnarray*}
Si la matrice $\bS_W$ est inversible, r\'esoudre 
ce probl\`eme revient \`a chercher $\bw$ le vecteur propre
associ\'e \`a $\lambda_{max}$ la plus grande valeur propre de la matrice 
$\bS_W^{-1}\bS_B$ :
$$
 \lambda_{max} \cdot \bw = \bS_W^{-1}\bS_B \cdot \bw .
$$
Remarquons que la matrice $\bS_B$ est de rang 1, ce qui implique
qu'il n'existe qu'une seule valeur propre. Dans le cas qui nous
int\'eresse on peut ais\'ement obtenir le vecteur propre en remarquant
que $S_B\cdot \bw$ est toujours dans la direction du vecteur $\bm_1-\bm_2$:
$$
\bw= \alpha  \bS_W^{-1} (\bm_1-\bm_2).
$$
avec $\alpha$ un scalaire tel que $\|\bw\|=1.$



\subsection{Cas g\'en\'eral}
%%%%%%%%%%%%%%%%%%%%%%%%%%%%%%%%%%%%%%%%


La g\'en\'eralisation du discriminant lin\'eaire de Fisher au cas multi-classes
recherche un espace vectoriel de dimension $p$ avec  $p<K-1$ telle que les
$K$ classes soient s\'epar\'es au mieux dans cet espace.
Le but de l'analyse factorielle discriminante consiste donc \`a trouver $p$
vecteurs $\bw_k$, qui d\'efinissent $p$ variables 
discriminantes :
$$
y_k= \bw_k \cdot \x,
$$
L'ensemble des vecteurs $\bw_k$, les facteurs, peut s'\'ecrire de mani\`ere
condens\'e sous la forme d'une matrice   $\bW$ de dimension $d$ par $p$ 
(chaque colonne de cette matrice  correspond \`a un des facteurs). 


Notons :
\begin{itemize}
\item 
$$
\tilde{\bm}_k = \frac{1}{n_k} \sum_{i=1}{n_k} \y_{ik}
              = \frac{1}{n_k} \sum_{i=1}{n_k} \y_{ik}=\bW^t \bm_k,
$$
les vecteurs moyennes des nouvelles variables ;
\item
$$
\tilde{\bm}  =\bW^t \bm,
$$
le vecteur moyenne de l'ensemble des vecteurs forme, exprim\'e en fonction des facteurs ;
\item
\begin{eqnarray*}
\tilde{\bS}_W &=& \sum_{k=1}^K \sum_{i=1}^{n_k} (\y_{ik}-\tilde{\bm}_k)(\y_{ik}-\tilde{\bm}_k)^t\\
             & =& \bW^t \bS_W \bW,
\end{eqnarray*}
la matrice de d'inertie intra-classes des nouvelles variables ;
\item
\begin{eqnarray*}
\tilde{\bS}_B & = & \sum_{k=1}^K n_k \cdot (\tilde{\bm}_k - \tilde{\bm})(\tilde{\bm}_k - \tilde{\bm})^t ,\\
      &= &   \bW^t \bS_B \bW,
\end{eqnarray*}
la matrice d'inertie inter-classes des nouvelles variables.
\end{itemize}

Pour obtenir les vecteur $\bw_k$ composant la matrice $\bW$, il est possible
de proc\'eder par construction. Le vecteur $\bw_1$, appel\'e premier facteur
discriminant, est la solution du probl\`eme :
$$
\left \{ \begin{array}{l}
\bw_1 = arg \max_{\boldv} \frac{\boldv^t \bS_B  \boldv}{\boldv^t \bS_W \boldv}  \\
\| \bw_1 \|=1    \\
\end{array}
\right.
$$ 
Dans la section pr\'ec\'edente nous avons vu que la solution de ce probl\`eme
\'etait le vecteur propre de $\bS_W^{-1}\bS_B$ associ\'e \`a la plus
grande valeur propre. Le second facteur, s'il existe, est d\'efini comme le 
vecteur propre associ\'e \`a la deuxi\`eme plus grande valeur propre.

En poursuivant cette construction, on trouve autant d'axes discriminants
que de valeurs propres non nulles,  c'est \`a dire au plus $K-1$, le
rang maximum de la matrice $\bS_W^{-1}\bS_B$. Remarquons que les vecteurs
propres ainsi trouv\'es sont $\bS_W$ orthogonaux :
$$
\bw_\ell^t \cdot \bS_W \cdot \bw_k = 0
$$
si $\bw_k$ et $\bw_\ell$ sont deux facteurs distincts.

Il semble aussi int\'eressant de savoir quel crit\`ere
optimise ce proc\'ed\'e de construction. \`A chaque \'etape,
on cherche un vecteur $\bw_k$ diff\'erent de tous  les vecteurs trouv\'es
pr\'ec\'edemment, qui maximise le crit\`ere 
\begin{eqnarray*}
J_k & = & \frac{\bw_k^t \bS_B  \bw_k}{\bw_k^t \bS_W \bw_k} \\
    & = & \bw_k^t  \bS_W^{-1}\bS_B \bw_k \\
    & = & trace \left [ \bw_k  \bw_k^t  \bS_W^{-1}\bS_B  \right ]
\end{eqnarray*} 
Ainsi \`a la fin de la construction, on peut affirmer que la
somme des crit\`eres $J_k$ est maximis\'ee (car chacun des termes de
la somme est maximis\'ee)  sous la contrainte
que les vecteurs $\bw_k$ sont orthogonaux entre eux. La matrice
$\bW$ est donc solution de :
$$
\left \{ \begin{array}{l}
\max \sum_{k=1}^p  trace \left [ \bw_k  \bw_k^t  \bS_W^{-1}\bS_B  \right ] = \max trace \left [ \bW^t \bS_W^{-1}\bS_B \bW \right ]         \\
\bW^t \cdot \bS_W \cdot \bW=I
\end{array}
\right.
$$
Ce crit\`ere met en \'evidence les liens existant avec l'analyse en composantes
principale.

Remarquons qu'il est possible de montrer que la matrice $\bW$ est aussi
solution de 
$$
\left \{ \begin{array}{l}
\max \frac{det\left ( \tilde{\bS}_B \right )}{det \left (\tilde{\bS}_W \right ) }=\frac{det\left ( \bW^t \bS_B \bW \right )}{det\left (\bW^t \bS_W \bW \right )}\\
\bW^t \bS_W \bW=I
\end{array}
\right.
$$
Ce dernier crit\`ere peut  s'interpr\`eter comme le rapport
 entre deux volumes. Ce rapport sera important lorsque  les 
classes projet\'ees  occuperons en moyenne un petit volume autour
de leur moyenne ($det(\tilde{\bS}_W)$ petit), et que les vecteurs moyennes
projet\'es occuperons occuperons un grand volume par rapport au pr\'ec\'edent.
Ceci correspond intuitivement bien \`a la notion de s\'eparation que l'on recherche.

Cette formulation du probl\`eme en terme de crit\`ere globale a
le d\'efaut d'admettre plusieurs solutions $\bW$ dont la solution
obtenue par construction. En effet toutes les transformation lin\'eaires
non singuli\`ere de $\bW$ laisse les crit\`eres invariants.

Notons que la matrice $\bW$ peut \^etre trouver de deux autre mani\`eres. Il 
est en effet direct de montrer que :
$$
arg \max_{\bw} \frac{\bw^t \bS_B \bw}{\bw^t \bS_W \bw} =
arg \max_{\bw} \frac{\bw^t \bS_T \bw}{\bw^t \bS_W \bw} =
arg \max_{\bw} \frac{\bw^t \bS_B \bw}{\bw^t \bS_T \bw}.
$$
En utilisant le processus de construction pr\'ec\'edent \`a partir
de ces crit\`eres, on trouve la m\^eme matrice $\bW$. Il s'ensuit
que les matrices  $(\bS_W^{-1}\cdot \bS_B)$, $(\bS_W^{-1}\cdot \bS_T)$ et 
$(\bS_T^{-1}\cdot \bS_B)$ ont m\^eme vecteurs propres.

%\subsection{Analyse en composante principale et analyse factorielle discriminante}

%L'analyse factorielle discriminante (AFD) peut s'interpr\`eter comme 
%une analyse en composante
%principale particuli\`ere. Il existe de nombreuses fa\c{c}on
%de pr\'esenter l'analyse en composantes principales (ACP). Rappelons bri\`evement
%l'une d'elle :

%soit un ensemble (dans notre cas $\R^d$) muni d'une m\'etrique
%$M$. L'inertie totale d'un nuage de point $\x_1,\cdots,\x_N$ est 
%d\'efinie par :
%\begin{equation}
%\label{eq:idem}
%I_T=\sum_{i=1}^N (\x_i-\bm)M(\x_i-\bm)^t=trace\left[ M \sum_{i=1}^N (\x_i-\bm)^t(\x_i-\bm)   \right]=trace \left [ M \bS_T  \right ] 
%\end{equation}
%L'analyse en composante principale  peut se d\'efinir comme la recherche d'un
%ensemble de facteurs, d\'efinissant de nouvelles variables, qui poss\`edent
%une variance maximum et qui sont orthogonaux par rapport \`a une m\'etrique 
%$M^{-1}$. 


%d'une sous espace vectoriel, sur lequel l'inertie projet\'ee
%soit maximum. Cela revient \`a chercher une matrice de projection
%$\bW$ telle que :
%$$
%\left \{ \begin{array}{l}
%\bW = arg \max_{\bV}trace \left [ \bV^t \cdot M \bS_T  \cdot \bV \right ] \\
%\bW^t \bW=I
%\end{array}
%\right.
%$$
%La solution bien connue de ce probl\`eme est 
%$\bW$ la matrice des vecteurs propres de $M \bS_T$. 

%Cette pr\'esentation permet de voir que l'analyse factorielle
%discriminante est une ACP de l'ensemble des vecteurs forme,
%r\'ealis\'ee avec la m\'etrique
%$\bS_W^{-1}$.


%D'apr\`es les \'egalit\'es \ref{eq:idem}, on d\'eduit que 
%l'AFD peut aussi s'interpr\`eter comme une ACP des
%vecteurs moyennes $\bm_1,\cdots,\bm_K$ avec la m\'etrique
%$\bS_T^{-1}$ ou $\bS_W^{-1}$. 

%L'impl\'ementation classique de l'AFD utilise cette derni\`ere relation :
%\begin{itemize}
%\item
%les vecteurs formes initiaux sont transform\'ees de mani\`eres \`a ce
%que les nouveaux vecteurs poss\`edent une matrice d'inertie 
%intra classe $\bS_W$ \'egale \`a l'identit\'e, 
%\item
%puis la matrice d'inertie inter-classe, calcul\'ee \`a partir des
%vecteurs transform\'es, est d\'ecompos\'ee sur sa base de vecteurs propres.
%\end{itemize}

%Notons enfin, que Gallinari {\em et al.} (1988) on  montr\'e que
%l'AFD entretient des relations \'etroites
%avec les r\'eseaux de neurones multi-couches poss\'edant une couche 
%cach\'ee et des fonctions d'activation lin\'eaires.



\section{Multidimensional scaling}

Pour r\'eduire la dimension des vecteurs forme, une alternative
consiste \`a utiliser des techniques de
multidimensionnal scaling. Ces techniques sont tr\`es
utilis\'ees en psychologie et sociologie, o\`u les objets
d'\'etude (les individus) ne sont pas d\'ecrit en terme de
caract\'eristiques individuelles mais les uns par 
rapport aux autres par la mesure de leur diff\'erence deux \`a 
deux. Ces  mesures sont appel\'ees dissimilarit\'es et 
non distances car elles ne v\'erifient pas l'in\'egalit\'e 
triangulaire. 

Les m\'ethodes de  multidimensional scaling visent
\`a repr\'esenter au mieux des objets dans un espace 
visualisable, de fa\c{c}on \`a ce que les distances
entre ces objets dans cet espace 
soient aussi proches que possible des dissimilarit\'es
initiales.

Les dissimilarit\'es mesurent les diff\'erences entre toutes
les paires d'objets. Ainsi les dissimilarit\'es entre $N$ objets
sont sp\'ecifi\'ees par une matrice $N \times N$,
$\delta=\{ \delta_{ij}\}_{i,j=1..N}$. Ce genre de matrices
peut avoir  diff\'erentes origines    :
\begin{itemize}
\item Le jugement humain peut souvent \^etre traduit par des
mesures de  dissimilarit\'es. Une personne amen\'ee \`a quantifier
la diff\'erence de confort entre deux voitures  pourra, par
exemple, choisir une chiffre entre 1 et 10 qui correspond 
\`a l'intensit\'e de la diff\'erence per\c{c}ue.
\item Les donn\'ees telles que les temps de transports entre
des paires de villes se pr\'esentent naturellement sous la forme
d'une matrice de dissimilarit\'es.
\item Enfin, notons qu'il est toujours possible de d\'eriver
une matrice de dissimilarit\'es 
(les distances sont des dissimilarit\'es) d'un ensemble de vecteurs
forme. Il suffit en effet de calculer les distance entre
tous les vecteurs forme. C'est cette derni\`ere approche
qui justifie la pr\'esence de cette section dans ces notes.
\end{itemize}

Parmi les techniques de positionnement multidimensionel, les
approches m\'etriques sont couramment oppos\'ees aux approches
non m\'etriques. Les premi\`eres produisent des repr\'esentations
pr\'eservant au mieux l'information quantitative \cite{Sammon1969}
contenue dans les donn\'ees, alors que les secondes privil\'egient 
l'information qualitative \cite{Kruskal1964}.    

\subsection{Projection de Sammon}

La projection de Sammon est une m\'ethode m\'etrique tr\`es
populaire. Cette m\'ethode cherche \`a ``projeter'' non
lin\'eairement les vecteurs formes $\x_1, \cdots, \x_N$ dans un espace de
plus faible dimension ($\R^1$, $\R^2$, ou bien $\R^3$). Notons
\begin{itemize}
\item
$\y_1, \cdots, \y_N$ les repr\'esentations des $\x_i$ dans cet
espace de faible dimension,
\item
$\delta_{ij}$, la distance (ou bien dissimilarit\'e) entre les vecteurs
formes $\x_i$ et $\x_j$,
\item
$d_{ij}$, la distance (euclidienne ou autre) entre  $\y_i$ et $\y_j$.  
\end{itemize}
Le crit\`ere propos\'e par \citeasnoun{Sammon1969} pour juger de la 
qualit\'e de la repr\'esentation est le suivant : 
\begin{equation}
S=\frac{1}{\sum_{i>j} \delta_{ij}} \sum_{i>j}\frac{(\delta_{ij}-d_{ij})^2}{\delta_{ij}},
\end{equation}
Remarquons que ce crit\`ere prend en compte les erreurs commises sur les
petites distances, contrairement \`a une ACP. Il est normalis\'e de mani\`ere
\`a \^etre invariant pour des rotations, translations et changements d'\'echelles.

La recherche de la configuration optimale des $\y_i$ peut s'effectuer
par une descente de gradient
$$
\y_k^{q+1}= \y_k^q -\alpha_q  \nabla_{y_k}^t S
$$
Par exemple, si $d_{ij}$ est la distance euclidienne,
le gradient du crit\`ere par rapport au point $\y_k$ prend la forme 
suivante :
$$
\nabla_{y_k}^t S = \frac{1}{\sum_{i>j} \delta_{ij}} \sum_{j \neq k} \frac{(d_{kj}-\delta_{kj})}{\delta_{kj}} \cdot \frac{(\y_{k}-\y_{j})}{d_{kj}} 
$$
La configuration initiale des $\y_i$ peut \^etre choisie au hasard mais le
processus semble converger plus vite si l'on part d'une solution 
approch\'ee (celle obtenue par ACP par exemple.)


\subsection{Projection de Kruskal}

L'approche non m\'etrique favorise les repr\'esentations qui 
privil\'egient l'ordre relatif entre les dissimilarit\'es initiales 
plut\^ot que les valeurs.  Reprenons les notations utilis\'ees pour
la projection de Sammon, en y ajoutant les $\hat{d}_{ij}$ qui sont
des nombres v\'erifiant la contrainte suivante : 
si les dissimilarit\'es initiales sont ordonn\'ees
$$
\delta_{i_1 j_1} \leq \cdots \leq \delta_{i_N j_N},
$$
alors on a
$$
\hat{d}_{i_1 j_1} \leq \cdots \leq \hat{d}_{i_N j_N}.
$$
\citeasnoun{Kruskal1964} propose de trouver une configuration des
$\y_i$ telle que les distances $d_{ij}=dist(\y_i,\y_j)$ soient
optimales au sens du crit\`ere
$$
S=\sqrt{\frac{S^*}{T^*}}=\sqrt{\frac{\sum_{i>j}(d_{ij}-\hat{d}_{ij})^2}{\sum_{i>j} d_{ij}^2}}
$$ 
L'optimisation de ce crit\`ere est plus d\'elicat que  celui de
Sammon. Remarquons d'ailleurs que bien que l'article de
Sammon soit post\'erieur de cinq ans, son crit\`ere et
la m\'ethode d'optimisation propos\'ee constituent un 
cas particulier de l'approche de Kruskal.

Kruskal propose une m\'ethode d'optimisation altern\'ee
pour optimiser les crit\`eres. Chaque it\'eration
se partage en deux \'etapes :
\begin{itemize}
\item une r\'egression isotonique est utilis\'ee pour calculer les
$\hat{d_{ij}}$ qui minimisent le crit\`ere (on consid\`ere les $d_{ij}$ fix\'es),
et respectent la contrainte de montonicit\'e.
\item le gradient du crit\`ere par rapport \`a chaque $\y_i$ est calcul\'e,
et une nouvelle configuration $\y_1, \cdot, \y_N$ est obtenue en modifiant
les points dans la direction du gradient (un pas d'une descente de gradient).
%Si $d_{ij}$ est la distance euclidienne, on aura par exemple :
%$$
%\nabla_{y_k}S= S \sum_{i>j} \delta_{ij}} \sum_{j \neq k} \frac{(d_{kj}-\delta_{kj})}{\delta_{kj}} 
%\cdot \frac{(\y_{k}-\y_{j})}{d_{kj}} 
%$$
\end{itemize}

\section{S\'election de caract\'eristiques}

Dans cette section sont pr\'esent\'ees quelques m\'ethodes qui permettent
de s\'electionner $q$ variables parmi les  $d$ disponibles. Le but
consiste \`a garder les variables les plus discriminantes, c'est-\`a-dire,
les variables qui pour un classifieur donn\'e vont produire le taux
d'erreur le plus petit possible. Ce probl\`eme peut \^etre partag\'e
en deux sous probl\`emes distincts :
\begin{itemize}
\item
d'une part, il faut disposer d'un crit\`ere pour juger la
qualit\'e d'un groupe de variables,
\item
d'autre part, il est n\'ecessaire d'utiliser des heuristiques
pour optimiser ce crit\`ere. En pratique, il est en effet souvent
impossible d'explorer tous les groupe des $q$ variables parmi $d$. 
\end{itemize}

\subsection{Crit\`eres de choix}
Le crit\`ere, qui mesure la performance d'un classifieur, est le
risque total $E^*$, qui est \'equivalent \`a la probabilit\'e d'erreur 
dans le cas o\`u le co\^ut $\{0,1\}$ est utilis\'e. Ce crit\`ere 
para\^{\i}t constituer un choix logique pour juger de la qualit\'e
d'un groupe de variables, mais en pratique, de nombreuses autres
mesures plus rapides \`a estimer ont \'et\'e propos\'ees :    
\begin{itemize}
\item
$C_1=trace \left [  \bS_{Wq}^{-1} \bS_{Bq} \right ]=\sum_{i=1}^q \lambda_i$. 
Ce crit\`ere vient de l'analyse factorielle discriminante. Plus 
il est grand et plus le groupe de $q$ variables consid\'er\'e est
discriminant au sens de l'AFD. C'est donc une mesure de la s\'eparabilit\'e
lin\'eaire para rapport au groupe de variables. 
\item
$C_2=\frac{|\bS_{Wq}|}{|\bS_{Tq}|}=\prod_{i=1}^{q} \frac{1}{1+\lambda_i}$ (lambda
de Wilks). Comme pr\'ec\'edemment ce crit\`ere peut s'interpr\'eter
dans le cadre de l'AFD. Remarquons que minimiser $C_2$ revient \`a
maximiser $C_1$.
\item 
$C_3=[(\bmu_1 - \bmu_2)^t \bSigma^{-1} (\bmu_1 - \bmu_2)]^{\frac{1}{2}}$
(distance de Mahalanobis). Dans le cadre limit\'e de la discrimination
lin\'eaire entre deux classes, l'erreur commise ne d\'epend que de cette
distance (Chapitre 2, exemple \ref{ex:erreur}). Notons
que dans ce cas de figure, $C_1$, $C_2$ et $C_3$ sont \'equivalents.
\item
$C_4=\sqrt{p_1 p_2} \int \sqrt{f_1(\x) f_2(\x) }d\x$ (distance de Bhattacharya).
Dans le  cas de deux classes, le risque de Bayes est born\'e sup\'erieurement
par cette distance :
\begin{eqnarray*}
E^* & = &   \E\left [ R(\hat{c}(X)|X) \right ]  \\
    & = & \int \min \left ( \pi(1 |\x) ,\pi(2 |\x) \right ) f(\x) d\x \\
    & = & \int \min \left ( p_1 f_1(\x) ,p_2 f_2(\x) \right ) d\x \\
    & \leq &  \sqrt{p_1 p_2} \int \sqrt{f_1(\x) f_2(\x) }d\x
\end{eqnarray*}
car $min(a,b)\leq a^s b^{1-s}$. Remarquons encore que dans le cas de la discrimination
lin\'eaire, ce crit\`ere est \'equivalent aux trois pr\'ec\'edents.
\end{itemize} 

\subsection{Proc\'edures de s\'election}

Les heuristiques de s\'election de variables comparent \`a chaque it\'eration 
deux groupes distincts par rapport \`a la valeur d'un crit\`ere, et
retiennent le meilleur groupe au sens de ce crit\`ere.

La s\'election ascendante commence par consid\'erer toutes les variables
s\'eparement. La meilleure variable (au sens du crit\`ere choisi) est 
s\'electionn\'ee lors de la premi\`ere \'etape. La seconde \'etape cherche
quelle variable ajout\'ee \`a la premi\`ere produit la meilleure valeur
du crit\`ere. Ce processus de construction est poursuivi jusqu'\`a obtenir
les $q$ variables souhait\'ees. Notons que cette heuristique est optimale
\`a chaque \'etape, mais ne garantit pas que le groupe de $q$ variables
final soit le meilleur. Ceci est vrai seulement si le crit\`ere peut se
d\'ecomposer en somme ou produit dont chaque terme ou facteur est
fonction d'une seule variable.

La s\'election descendante commence par consid\'erer toutes les variables,
et les \'elimine une par une. La variable \'elimin\'ee \`a chaque \'etape
est \'evidemment la moins bonne au sens du crit\`ere. 


% rajouter un passage sur branch and bound
