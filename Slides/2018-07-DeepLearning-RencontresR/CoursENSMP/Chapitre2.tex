\chapter{Approche statistique et mod\`eles param\'etriques}
\nocite{Flury1994}
Dans le chapitre pr\'ec\'edent nous avons vu que la r\`egle de Bayes
qui permet de d\'efinir un classifieur optimal au sens de la
minimisation conjointe des probabilit\'es d'erreur et de doute,
n\'ecessite la connaissance des loi {\em a posteriori}
$\pi(\x | k)$. Malheureusement dans les probl\`emes r\'eels de 
reconnaissance des formes, le mod\`ele probabiliste que suivent
les donn\'ees est inconnu. 

Dans le cadre de l'approche statistique (``sampling paradigm''), les
densit\'es consid\'er\'ees sont les densit\'es conditionnelles 
aux classes $f_k(\x | \theta_k)$. En effet, la connaissance de ces densit\'es
conjointement aux proportions des classes $p_k$,  suffit \`a appliquer
la r\`egle de Bayes car :
$$
p_k \cdot f_k(\x|\theta_k) \propto \pi(k | \x).
$$
Une proc\'edure possible pour d\'efinir le classifieur consiste \`a
consid\'erer que les densit\'es conditionnelles aux classes appartiennent
\`a une famille de densit\'es d\'efinies par peu de param\`etres. Cette
approche param\'etrique comporte alors deux \'etapes :
\begin{enumerate}
\item le choix d'un mod\`ele ;
\item l'estimation des param\`etres de ce mod\`ele.
\end{enumerate}
Cette d\'emarche revient \`a approximer les densit\'es {\em a posteriori} par
$$
\hat{\pi}(k| \x)=\frac{\hat{p}_k \cdot f_k(\x | \hat{\theta}_k)}{\sum_{\ell=1}^{K} \hat{p}_k \cdot f_\ell(\x | \hat{\theta}_\ell)}.
$$  

Ce chapitre pr\'esente dans une premi\`ere partie quelques  mod\`eles 
couramment utilis\'es en reconnaissance statistique des formes.
La seconde section  expose l'estimation par maximum de 
vraisemblance des param\`etres des densit\'es de classe. L'estimation
par maximum de vraisemblance n'est naturellement pas la seule 
solution existante, qui permet de r\'esoudre les probl\`emes 
d'estimation rencontr\'es en discrimination, mais elle a le 
m\'erite d'\^etre relativement simple. Le lecteur int\'eress\'e
pourra consulter l'excellent livre de B. Ripley(1996) pour une introduction
\`a d'autres approches.   Enfin,
la derni\`ere section envisage la d\'elicate question du choix de  mod\`ele.

\section{La loi normale multidimensionnelle}

% presentation de la loi
 % combinaison lineaires
La loi normale repose sur les travaux de Jacques Bernouilli (1654-1705).
Parfois attribu\'e \`a Laplace et Gauss, elle tient son nom
de Pearson (le p\`ere). F. Galton et K. Pearson utilisaient d\'ej\`a
la loi normale en dimension 2 \`a la fin du 19\ieme si\`ecle et l'extension
au cas g\'en\'eral a \'et\'e r\'ealis\'e durant le premier quart
du 20\ieme si\`ecle. 

Dans le cadre de la reconnaissance des formes, la loi
normale est couramment utilis\'ee comme densit\'e 
de classe : on suppose ainsi que chaque $\x_i \in \R^d$  
appartenant \`a la classe ${\cal C}_k$ suit une loi de densit\'e 
\begin{equation}
f_k(\x_i | \theta_k)=(2\pi)^{-\frac{d}{2}}  \det{|\bSigma_k|}^{-\frac{1}{2}}
\exp{(-\frac{1}{2}(\x-\bmu_k)^t \bSigma_k^{-1} (\x-\bmu_k))} 
\end{equation}
avec $\bmu_k$  le vecteur moyenne, $\bSigma_k$ la matrice 
de variance covariance, et $d$ la dimension des vecteurs $\x_i$.
De mani\`ere plus concise, on note 
$$
\x_i \sim {\cal N}_d(\bmu_k,\bSigma_k)
$$

D'un point de vue g\'eom\'etrique (si l'on se place dans $\R^d$), faire
l'hypoth\`ese de normalit\'e revient \`a supposer que tous les vecteurs
forme d'une classe $k$ donn\'ee appartiennent \`a une hyperellipso\"{\i}de 
de centre $\bmu_k$ avec une certaine probabilit\'e $\alpha$. L'\'equation
de l'hyperellipso\"{\i}de est donn\'ee par :
$$
r_\alpha^2=(\x_i-\bmu_k)^t \bSigma_k (\x_i-\bmu_k)
$$ 
Si l'on d\'ecompose  la matrice de covariance dans sa base de vecteurs propres :
\begin{equation}
\bSigma_k=\lambda_k \cdot \Dbold_k \cdot \Bbold_k \cdot \Dbold_k^t
\end{equation}   
il est possible d'interpr\'eter g\'eom\'etriquement les quantit\'es
mises en \'evidence :
\begin{itemize}
\item $\lambda_k=\det{|\bSigma_k|^{\frac{1}{d}}}$ est interpr\'et\'e comme le volume de
la $k$\ieme classe. En effet plus $\lambda_k$ est grand plus la classe occupera une
place importante dans l'espace $\R^d$. Cette notion ne doit pas \^etre confondue avec 
le nombre d'individus de la classe qui est relatif \`a la proportion $p_k$; ce n'est pas
parce qu'une classe occupe un grand volume qu'elle contient forc\'ement beaucoup d'individus.

\item $\Bbold_k$ (avec $det{|B_k|}$) est la matrice diagonale des valeurs propres.
Elle caract\'erise la forme de la
classe $k$. Plus une valeur propre est importante plus l'enveloppe de la classe
est ``allong\'ee'' dans la direction du vecteur propre correspondant.

\item $\Dbold_k$ est la matrice des vecteurs propres, qui sont les axes principaux
de l'hyperellipso\"{\i}de.  Cette matrice donne l'orientation de la 
classe $k$. C'est une matrice orthogonale de changement de base. Par rapport
aux axes de r\'ef\'erence, la base de vecteurs propres est obtenue par rotation. 
\end{itemize}

\begin{figure}[hbtp]
\begin{center}
\leavevmode
\hbox
{
\epsfxsize=\textwidth
\divide\epsfxsize by 100
\multiply\epsfxsize by 50
\epsffile{/home/soleil/ambroise/figures/gauspar.eps}
}
\end{center}
\caption{Param\'etrisation de la matrice de covariance dans le cas
bidimensionnel}
\label{fig :param}
\end{figure}

La matrice de covariance est sym\'etrique d\'efinie positive. Dans
$\R^d$, une matrice de covariance repr\'esente  $d \cdot \frac{(d+1)}{2}$ param\`etres \`a estimer pour une seule classe. Si l'on se place dans $\R^8$, 
cela fait 144  param\`etres \`a estimer pour une
seule matrice de covariance.  
Un moyen simple et classique de r\'eduire ce nombre consiste \`a faire
l'hypoth\`ese (forte), que toutes les classes ont m\^eme matrice 
de covariance $\bSigma$. Ainsi classiquement deux cas d'\'etudes
sont distingu\'es :
\begin{itemize}
\item h\'et\'erosc\'edasticit\'e : matrice de covariance propre \`a chaque classe ($\bSigma_k$).
\item homosc\'edasticit\'e : matrice de covariance commune \`a toutes les classes ($\bSigma_k=\bSigma$).
\end{itemize}

\subsection{Homosc\'edasticit\'e}
Lorsque les matrices de covariance sont identiques pour toutes les
classes, cela revient \`a supposer que les vecteurs forme de chaque
classe tombent dans des hyperellipso\"{\i}des de m\^eme volume,
m\^eme forme et m\^eme orientation. Dans ce cas la r\`egle de
Bayes choisit la classe $\hat{c}(\x)$ telle que :
\begin{eqnarray*}
\hat{c}(\x) & = & arg \max_k \pi (k | \x ),\\
            & = & arg \max_k (p_k \cdot f_k(\x)),\\
            & = & arg \min_k -2 \cdot \log p_k  + (\x - \bmu_k)^t \bSigma^{-1} (\x - \bmu_k).
\end{eqnarray*}

Remarquons que si les proportions $p_k$ sont toutes \'egales, alors cette r\`egle de
d\'ecision revient \`a affecter un vecteur forme \`a la classe la plus proche au
sens de la distance de \citeasnoun{Mahalanobis1936} :
$$
\delta(\x,\bmu_k)=[(\x - \bmu_k)^t \bSigma^{-1} (\x - \bmu_k)]^{\frac{1}{2}}.
$$ 
Si, de plus, la matrice de covariance est proportionnelle \`a la matrice identit\'e,
$$
\bSigma=\sigma \cdot I,
$$
alors la distance de Mahalanobis est \'equivalente \`a la distance euclidienne. Dans ce dernier
cas les classes sont suppos\'ees avoir une forme sph\'erique et un volume $\sigma$.
Lorsque les proportions sont diff\'erentes, le terme $-2 \cdot \log p_k$ biaise la 
d\'ecision en faveur de la classe la plus probable {\em a priori}.


La r\`egle de d\'ecision peut s'exprimer sous une forme plus simple lorsque le terme
quadratique est d\'evelopp\'e, car $\x^t \bSigma^{-1} \x$ est une expression ind\'ependante
de l'indice de classe : 
\begin{eqnarray*}
\hat{c}(\x) & = & arg \max_k (\bSigma^{-1} \bmu_k)^t \cdot \x + (-\frac{1}{2}{\bmu_k}^t\bSigma^{-1} \bmu_k + \log p_k),\\
            & = & arg \max_k \bw_k^t \cdot \x + w_{k0}.
\end{eqnarray*}

La fonction de d\'ecision est lin\'eaire et on parle d'analyse discriminante lin\'eaire, ce qui
implique, que les fronti\`eres s\'eparant deux  r\'egions voisines de d\'ecision, sont des hyperplans.
Consid\'erons  ${\cal R}_k$ et ${\cal R}_\ell$ deux r\'egions contig\"ues :  la fronti\`ere entre ces
deux r\'egions est d\'ecrite par l'\'equation : 
$$
(\bSigma^{-1} (\bmu_k-\bmu_\ell))^t \cdot (\x - \x_0)=0,
$$ 
o\`u 
$$
\x_0=\frac{1}{2}(\bmu_k + \bmu_\ell) - \log (\frac{p_k}{p_\ell}) \frac{(\bmu_k-\bmu_\ell)}{(\bmu_k-\bmu_\ell)^t\bSigma^{-1} (\bmu_k-\bmu_\ell) }.
$$
Ainsi la surface s\'eparatrice est un hyperplan orthogonal \`a $\bSigma^{-1} (\bmu_k-\bmu_\ell)$ et
passant par le point $\x_0$.

Notons que dans le cas particulier o\`u les proportions 
sont \'egales et la matrice de covariance proportionnelle \`a la matrice identit\'e, alors
l'hyperplan est orthogonal \`a l'axes reliant les vecteur moyennes $\bmu_k$ et $\bmu_\ell$
et le point $\x_0$ est exactement au milieu du segment d\'efini par $\bmu_k$ et $\bmu_\ell$.
Si les proportions sont diff\'erentes cela revient \`a translater l'hyperplan vers vecteur
moyenne de la classe la moins probable.

\begin{ex}\cite{Ripley1996}(suite de l'exemple \ref{ex:erreur})
Probabilit\'e d'erreur dans le cas de deux classes gaussiennes sous hypoth\`ese d'homosc\'edasticit\'e :
la r\`egle de d\'ecision prend la forme suivante
$$
\hat{c}(\x)=
\left \{ \begin{array}{l}
1 \ si \ A=(\bmu_1 - \bmu_2)^t \bSigma^{-1}(\x - \frac{1}{2}(\bmu_1 + \bmu_2)) > \log \frac{p_1}{p_2} , \\
2 \ sinon.\\
\end{array}
\right .
$$
Si $X$ appartient \`a la classe 1 alors on peut montrer que
$$
A \sim {\cal N}(\frac{1}{2}\delta^2,\delta^2)
$$
avec $\delta=[(\bmu_1 - \bmu_2)^t \bSigma^{-1} (\bmu_1 - \bmu_2)]^{\frac{1}{2}}$.
De m\^eme si $X$ appartient \`a la seconde classe alors 
$$
A \sim {\cal N}(-\frac{1}{2}\delta^2,\delta^2)
$$

Maintenant,  la probabilit\'e d'erreur peut s'\'ecrire comme : 
\begin{eqnarray*}
P(\mbox{Erreur})& = &  P(\X \in {\cal R}_2 | C=1) \cdot p_1 +  
P(\X \in {\cal R}_1 | C=2) \cdot p_2,\\
& = &  p_1 \cdot P(A\leq \log \frac{p_1}{p_2}| C=1) + p_2 \cdot  P(A > \log \frac{p_1}{p_2}| C=2),\\
                & = &  p_1 \cdot \Phi(-\frac{1}{2} \delta + \frac{1}{\delta} \log \frac{p_1}{p_2}) + p_2 \cdot \Phi(-\frac{1}{2} \delta - \frac{1}{\delta} \log \frac{p_1}{p_2})
\end{eqnarray*}


\end{ex}






\subsection{H\'et\'erosc\'edasticit\'e}
Dans le cas g\'en\'eral, chaque matrice de covariance est diff\'erente. 
La r\`egle de d\'ecision prend une forme plus complexe que dans le cas
pr\'ec\'edent :
\begin{eqnarray*}
\hat{c}(\x) & = & arg \max_k \pi (k | \x ),\\
            & = & arg \min_k -2 \cdot \log p_k  + \log(\det{|\bSigma_k|}) + (\x - \bmu_k)^t \bSigma_k^{-1} (\x - \bmu_k).
\end{eqnarray*}

Cette r\'egle est quadratique et les surfaces de d\'ecisions g\'en\'er\'ees
sont des hyperquadratiques : hypersph\`eres, hyperellispo\"{\i}de, 
hyperparabolo\"{\i}de, hyperhyperbolo\"{\i}de. L'\'equation g\'er\'erale
d'une  surface s\'eparatrice est alors de la forme :
$$
\x^t W_k \x + \bw_k^t \x + \bw_{k0}=0. 
$$  


 

\subsection{Mod\`ele parcimonieux}

\begin{table}\label{tab:modeles}
\begin{center}
\begin{tabular}{|c|c|}
\hline
mod\`ele & nombre de param\`etres \\
\hline
$[\lambda D B D']$&$\alpha+\beta$\\
$[\lambda_k D B D']$&$\alpha+\beta+K-1$\\
$[\lambda D B_k D']$&$\alpha+\beta+(K-1)(d-1)$\\
$[\lambda_k D B_k D']$&$\alpha+\beta+(K-1)d$\\
$[\lambda D_k B D_k']$&$\alpha+K\beta-(K-1)d$\\
$[\lambda_k D_k B D_k']$&$\alpha+K\beta-(K-1)(d-1)$\\
$[\lambda D_k B_k D_k']$&$\alpha+K\beta-(K-1)$\\
$[\lambda_k D_k B_k D_k']$&$\alpha+K\beta$\\
     & \\
$[\lambda A]$&$\alpha+d$\\
$[\lambda_k A]$  & $\alpha+d+K-1$\\
$[\lambda A_k$] &$\alpha+Kd-K+1$\\
$[\lambda_k A_k]$&$\alpha+Kd$\\
      &  \\
$[\lambda I]$&$\alpha+1$\\
$[\lambda_k I]$&$\alpha+d$\\
\hline
\end{tabular}
\end{center}
\caption{Nombre de param\`etres \`a estimer pour chacun des quatorze mod\`eles.
Lorsque un param\`etre diff\'erent est caclul\'e par classe, celui-ci est
indic\'e. Si le param\`etre est commun a toutes les classes, il ne porte
pas d'indice. Nous avons  $\alpha=Kd$ et  $\beta=\frac{d(d+1)}{2}$}
\end{table}

Une  solution interm\'ediaire  pour r\'eduire le nombre de param\`etres
\`a estimer consiste \`a consid\'erer la param\'etrisation
de la matrice de covariance. Cette param\'etrisation permet
de mettre en \'evidence des param\`etres qui poss\`edent une
signification g\'eom\'etrique et l'on peut alors autoriser
certains de ces param\`etres a \^etre commun \`a toutes les
classes et estimer les autres par classe. Cette approche
englobe bien \'evidemment les deux solutions pr\'esent\'ees
pr\'ec\'edemment et permet  de d\'efinir 8 mod\`eles
diff\'erents selon que l'on   autorise la libert\'e par classe 
des param\`etres de volume $\lambda_k$, de forme $\Bbold_k$, ou
bien d'orientation $\Dbold_k$. Dans le contexte du classement,
les mod\`eles parcimonieux ont \'et\'e exploit\'es par Flury {\em et al.}
(1994) et apppliqu\'es entre autre par \citeasnoun{Celeux1995} \`a la 
classification. 

Pour diminuer encore le nombre de param\`etres \`a estimer,
il est possible d'envisager des hypoth\`eses suppl\`ementaires
sur les matrices de covariance. Deux situations semblent
int\'eressantes :
\begin{itemize}
\item imposer le fait que les matrices de covariance sont diagonales. 
Dans ce cas, 
$$
\bSigma_k= \lambda_k \cdot \Abold_k
$$ 
avec  $\det(\Abold_k)=1$. Cette hypoth\`ese fournit 4 mod\`eles 
suppl\'ementaires ; 
\item faire l'hypoth\`ese que les matrices de covariances sont
proportionnelles \`a la matrice identit\'e :
$$
\bSigma_k= \lambda_k \cdot I,
$$
ce qui ajoute encore deux mod\`eles.
\end{itemize}  


Finalement, nous avons quatorze mod\`eles diff\'erent qui vont
du plus simple forcant toutes les matrices de covariance \`a \^etre
proportionnelles \`a la matrice identit\'e et \`a avoir m\^eme volume,
au plus compliqu\'e, qui n\'ecessite le calcul d'une matrice de 
covariance diff\'erente par classe. Le tableau \ref{tab:modeles}
donne le nombre de param\`etre \`a estimer pour chacun des quatorzes
mod\`eles.



\section{La loi de Student}
% Loi de Student
\citeasnoun{Ripley1996} remarque que la loi normale poss\`ede des ``queues''
tr\`es aplaties contrairement au distributions (empiriques) observ\'ees
dans des probl\`emes r\'eels. La loi de Student, qui poss\`ede des ``queues''
plus lourdes, peut donc \^etre utilis\'ee avantageusement.

La loi de Student multivariable peut \^etre d\'efinie par la densit\'e suivante
lorsque $\nu$ le nombre de degr\'e de libert\'e est sup\'erieur \`a deux
\begin{equation}
f_k(\x| \theta_k)=\frac{\Gamma(\frac{1}{2}(\nu+d))}{(\nu\pi)^\frac{d}{2} \Gamma(\frac{1}{2}\nu)}|\bSigma_k|^{-\frac{1}{2}}[1+\frac{1}{\nu}(\x-\bmu_k)^t\bSigma_k^{-1}(\x-\bmu_k)]^{-\frac{1}{2}(\nu+d)} 
\end{equation}

avec $\bmu_k$ le vecteur moyenne et $\bSigma_k$ la matrice 
d'\'echelle. La matrice de covariance est $\frac{\nu\bSigma}{\nu-2}$.

La r\`egle de d\'ecision prend la forme
\begin{eqnarray*}
\hat{c}(\x) & = & arg \min_k \frac{\nu+d}{2}\log[1+\frac{1}{\nu}(\x-\bmu_k)^t\bSigma_k^{-1}(\x-\bmu_k)]+\frac{1}{2}\log|\bSigma_k|-\log p_k
\end{eqnarray*}

\section{Estimation par maximum de vraisemblance}

Entre autres contributions \`a la statistique, R. Fisher (1890--1962) a
introduit le concept de vraisemblance en 1912 dans un article intitul\'e
``On absolute criterion for fitting frequency curves'' \cite{Fisher1912}.
Aujourd'hui (1997), les estimateurs du maximum de vraisemblance jouent un 
r\^ole central dans la th\'eorie de l'estimation.

\begin{defi}
Soit la r\'ealisation d'un \'echantillon i.i.d. ${\cal X}=(\x_1,...,\x_N)$ d'une 
variable al\'eatoire  de densit\'e  $f$ d\'ependant d'un param\`etre $\theta$. 
On note $f({\cal X} | \theta)=\prod_{i=1}^N f(\x_i | \theta)$ la densit\'e de 
l'\'echantillon et $\ell(\theta; {\cal X})=f({\cal X} | \theta)$ la vraisemblance 
du  param\`etre $\theta$.   
\end{defi}

La notion de vraisemblance am\`ene \`a r\'e\'ecrire la densit\'e de 
l'\'echantillon en consid\'erant le param\`etre $\theta$ comme fonction
d'un \'echantillon observ\'e, ${\cal X}$. Cette d\'efinition du concept de 
vraisemblance conduit naturellement la d\'efinition d'estimateur du
maximum de vraisemblance :

L'estimateur du maximum de vraisemblance $\hat{\theta}_{MV}$ de $\theta$ est
tel que $\hat{\theta}_{MV}=arg \max_{\theta}\ell(\theta;{\cal X})$.
Il est souvent plus avantageux de maximiser la log-vraisemblance $L(\theta; {\cal X})=\log \ell(\theta; {\cal X})$
plut\^ot que la vraisemblance. 

Dans le cas particulier o\`u la log-vraisemblance est deux fois diff\'erentiable et
le param\`etre est un scalaire,
$\hat{\theta}_{MV}$ est une solution du syst\`eme : 
\[
\left \{
\begin{array}{c}
\frac{\partial L(\theta; {\cal X})}{\partial \theta}=0 \\
\frac{\partial^2 L(\theta; {\cal X})}{\partial \theta^2}<0 \\
\end{array}
\right .
\]
Si le param\`etre \`a estimer est vectoriel, l'estimateur du maximum de
vraisemblance annule le gradient, et induit une matrice hessienne
d\'efinie n\'egative.  Notons que ces conditions sont  
n\'ecessaires mais pas suffisantes.


\subsection{Discrimination et maximum de vraisemblance}
Dans un probl\`eme de discrimination, l'on dispose d'un ensemble 
d'apprentissage 
$$
{\cal F}=\{(\x_1,c(\x_1)),\cdots,(\x_N,c(\x_N))\}
$$
Notons $n_k$ le nombre de vecteurs forme de la classe $k$ et 
$\x_{i,k}$ un vecteur forme de cette classe. La vraisemblance
de l'ensemble des param\`etres $(\theta,(p_1,\cdots p_{K}))$ du mod\`ele
connaissant ${\cal F}$ s'\'ecrit
\begin{eqnarray*}
\ell(\theta,(p_k); {\cal F})) & = & \prod_{k=1}^K \prod_{i=1}^{n_k} P(\X_{ik}=\x_{i,k},C=k;\theta),\\
                  & = & \prod_{k=1}^K \prod_{i=1}^{n_k} p_k \cdot f_k(\x_{i,k};\theta)
\end{eqnarray*}
Les estimateurs du maximum de vraisemblance de $(\theta,(p_1,\cdots p_{K-1}))$ 
maximisent 
\begin{equation}
L(\theta,(p_k); {\cal F}))  = \sum_{k=1}^{K} \sum_{i=1}^{n_k} \log f_k(\x_{i,k};\theta) + \sum_{k=1}^{K} n_k \log p_k.
\end{equation}
Si nous cherchons dans un premier temps les estimateurs des proportions, 
en prenant en compte la contrainte $\sum_{k=1}^{K} p_k =1$ grace \`a
un multiplicateur de Lagrange, alors on trouve
$$
\widehat{p}_k=\frac{n_k}{\sum_{\ell=1}^{K} n_{\ell}}=\frac{n_k}{N}
$$ 

\subsection{Les param\`etres de la loi normale}
Dans le cas o\`u les densit\'e de classe sont normales, 
les estimateurs du maximum  de vraisemblance s'\'ecrivent :
\begin{itemize}
\item pour les vecteurs moyennes 
$$
\widehat{\bmu}_k=\frac{\sum_{i=1}^{n_k}\x_{i,k}}{n_k},
$$
\item et pour les matrices de covariance
$$
\widehat{\bSigma}_k=\frac{1}{n_k} \sum_{i=1}^{n_k} (\x_{i,k}-\widehat{\bmu}_k)(\x_{i,k}-\widehat{\bmu}_k)^t.
$$
Remarquons que cet estimateur est biais\'e et qu'en pratique ce sont
les estimateurs d\'ebiais\'es qui sont couramment utilis\'es
$$
\widehat{\bSigma}_k^*=\frac{1}{n_k-1} \sum_{i=1}^{n_k} (\x_{i,k}-\widehat{\bmu}_k)(\x_{i,k}-\widehat{\bmu}_k)^t
$$
\end{itemize}

Si toutes les matrices de covariance des densit\'e de classe sont
suppos\'ees identiques (homosc\'edasticit\'e) alors
$$
\widehat{\bSigma}= \sum_{k=1}^{K} \frac{n_k}{N}\widehat{\bSigma}_k
$$
Ce dernier estimateur est aussi biais\'e, car l'estimateur du vecteur moyenne 
est utilis\'e \`a la place du vecteur moyenne  et l'on consid\`ere en g\'en\'eral
$$
\widehat{\bSigma}^*= \sum_{k=1}^{K} \frac{n_k}{N-K}\widehat{\bSigma}_k
$$
qui est sans biais.


Lorsque la matrice de covariance est d\'ecompos\'ee sur sa base
de vecteurs propres (mod\`eles parcimonieux), et que certains param\`etres 
sont fix\'es le calcul des estimateurs du maximum de vraisemblance se
complique l\'eg\`erement et n\'ecessite parfois l'usage
d'algorithmes it\'eratifs. Le lecteur int\'eress\'e consultera
avantageusement Flury {\em et al.}(1994).


\subsection{Les param\`etres de la loi de Student}
Dans le cas o\`u les densit\'es de classe sont de lois de Student
multivari\'ees, le calcul des estimateurs de maximum de vraisemblance
devient it\'eratif. Notons $Q_{ik}$ le terme quadratique
$$
Q_{ik}=(\x_{ik}-\bmu_k)^t\bSigma_k^{-1}(\x_{ik}-\bmu_k), 
$$
et $w_{ik}$ des coefficient de la forme
$$
w_{ik}=\frac{1}{1+Q_{ik}/\nu}.
$$
En utilisant ces notations, les estimateurs de maximum de
vraisemblance sont
\begin{itemize}
\item pour les vecteurs moyennes 
$$
\widehat{\bmu}_k=\frac{\sum_{i=1}^{n_k} w_{ik}\x_{i,k}}{\sum_{i=1}^{n_k} w_{ik }},
$$
\item et pour les matrices $\bSigma_k$ (qui ne sont pas les matrices de covariance,
mais des matrices d'\'echelles). 
)
$$
\widehat{\bSigma}_k=\frac{1}{n_k}\frac{\nu+p}{\nu} \sum_{i=1}^{n_k} w_{ik} (\x_{i,k}-\widehat{\bmu}_k)(\x_{i,k}-\widehat{\bmu}_k)^t.
$$
\end{itemize}
Comme les coefficients $w_{ik}$ sont fonctions des param\`etres que l'on
d\'esire estimer, les estimateurs du maximum de vraisemblance ne peuvent
\^etre d\'etermin\'es directement. 

Remarquons que les estimateurs produits sont des versions pond\'er\'ees
des estimateurs trouv\'es dans le cas gaussien. Les pond\'erations
donnant moins d'importance aux vecteurs forme isol\'es. 


\section{Choix de mod\`ele}

Lorsque un statisticien choisit un mod\`ele, il sait pertinemment
qu'il n'est qu'une approximation pratique de la r\'ealit\'e. 
Si la question n'est pas de trouver le mod\`ele id\'eal que
suivent les donn\'ees, il semble par contre int\'eressant de 
d\'eterminer quel est le meilleur mod\`ele candidat parmi un
certain nombre possible. Par exemple, on peut se poser la question
de savoir quel est le mod\`ele normal,  parmi les quatorze
mod\`eles parcimonieux, qui rend le mieux compte de la densit\'e
d'une classe. On pourrait s'attendre \`a ce que  le mod\`ele le plus 
compliqu\'e (celui qui poss\`ede le plus de param\`etres) donne
le meilleur r\'esultat. Cette derni\`ere observation  se confirme  
lorsque l'ensemble d'apprentissage comporte de tr\`es nombreux
exemples, mais ce n'est pas toujours le cas en pratique. 

Mais comment \'evaluer les performances d'un mod\`ele ? Dans le contexte
de la discrimination, deux approches sont envisageable :  
\begin{itemize}
\item une premi\`ere solution consiste \`a envisager la question sous l'angle
de la performance du classifieur produit par un mod\`ele donn\'e.
 Le mod\`ele retenu
sera celui qui engendre la plus petite probabilit\'e d'erreur de 
classement.  Cette perspective r\'eduit alors le probl\`eme \`a la 
question trait\'ee dans la derni\`ere section du premier chapitre ;

\item une seconde alternative consiste \`a consid\'erer un crit\`ere
d'ajustement du mod\`ele \`a la r\'ealit\'e, classe par classe. Ainsi, 
le mod\`ele le mieux ajust\'e devrait produire un classifieur le 
plus proche possible du classifieur bay\'esien id\'eal (celui qui
classifie connaissant les densit\'es r\'eelles) et donc, indirectement
minimiser l'erreur de classement. La fin de ce chapitre pr\'esente deux strat\'egies
diff\'erentes permettant de choisir entre plusieurs mod\`eles.
\end{itemize}      


\subsection{P\'enalisation de la vraisemblance}

Comment choisir entre plusieurs mod\`eles de diff\'erentes complexit\'es
lorsque la m\'ethode du maximum de vraisemblance est utilis\'e
comme strat\'egie d'estimation ? En r\`egle g\'en\'erale, sur 
un ensemble d'apprentissage donn\'e, le mod\`ele le plus
complexe donne la plus grande vraisemblance. Une fa\c{c}on
d'aborder le  probl\`eme du choix consiste \`a tenter de 
r\'epondre \`a la question suivante : comment se comporterait
le mod\`ele en moyenne  si l'on jugeait ses performances avec d'autres
donn\'eees ?
L'id\'ee sous-jacente est qu'un mod\`ele tr\`es compliqu\'e,
ajust\'e en utilisant un petit ensemble d'apprentissage,
risque de mal se comporter si l'on calcule ses performances
sur un ensemble test, car le mod\`ele sera ``sp\'ecialis\'e''
pour rendre compte du petit ensemble d'apprentissage.  

% definition de theta0 a partir de la distance de Kulback

La vraisemblance est une fonction d'un vecteur de param\`etre
$\theta$, pour une certaine r\'ealisation d'un \'echantillon.
 Si l'on \'ecrit la log-vraisemblance en 
fonction d'un \'echantillon de taille $N$ :
$$
L_N(\theta)=\sum_{i=1}^N \log f(\X_i ; \theta)
$$
alors d'apr\`es la loi forte des grands nombres, 
$$
\lim_{N \rightarrow \infty} L_N(\theta)/N= \int f(\x) \log f(\x ; \theta) d\x.
$$
Cette esp\'erance poss\`ede souvent un maximum unique $\theta_0$,
qui peut \^etre interpr\'et\'e comme la valeur de $\theta$
qui rend la densit\'e $f(\x ; \theta)$ aussi proche
que possible de la densit\'e vraie $f(\x)$ au sens de la distance
de Kullback :
$$
d(f,f_\theta)=\int f(\x) \log \frac{f(\x)}{f(\x ; \theta)}d\x.
$$

Sous certaines conditions, on montre que $\hat{\theta}$ l'estimateur
du maximum de vraisemblance converge en loi vers  $\theta_0$. 



Consid\'erons maintenant l'esp\'erance de l'\'ecart au mod\`ele, pour
un seul vecteur forme :
$$
D=2 \E[\log f(\X) - \log f(\X ; \hat{\theta})]
$$
Cette valeur moyenne est int\'eressante car elle mesure l'ajustement
moyen du mod\`ele en prenant en compte d'autre vecteurs formes que ceux
de ${\cal F}$. Le th\'eor\`eme suivant montre comment l'on peut
estimer cette esp\'erance et l'utiliser pour choisir le mod\`ele
le plus adapt\'e. 

\begin{th}\label{th:penalty}\cite{Ripley1996}
$$
2 \E[\log f(\X) - \log f(\X ; \theta_0)]=2 \E[\log f(\X) - \log f(\X ; \hat{\theta})] + \frac{1}{N} trace[KJ^{-1}] + O(1/\sqrt{N})
$$
avec 
\begin{itemize}
\item
$$
J=-\E[\frac{\partial^2 f(\X_i ; \theta_0)}{\partial \theta \partial \theta^T}]
$$
\item
$$
K=var[\frac{\partial f(\X_i ; \theta_0)}{\partial \theta}]
$$
la matrice d'information de Fisher.
\end{itemize}
\end{th}

Si l'on remplace esp\'erance par moyenne empirique dans le th\'eor\`eme \ref{th:penalty},
alors on trouve
\begin{equation}
\label{eq:complexite}
2 \sum_{i=1}^{N} \log \frac{f(\X_i)}{f(\X_i ; \theta_0)} \approx 2 \sum_{i=1}^{N} \log \frac{f(\X_i)}{f(\X_i ; \hat{\theta})}
+ trace[KJ^{-1}].
\end{equation}
Cette \'equation montre que la d\'eviance de $\hat{\theta}$ calcul\'ee sur 
${\cal F}$
$$
D_{\cal F}(\hat{\theta})=2 \sum_{i=1}^{N} \log \frac{f(\X_i)}{f(\X_i ; \hat{\theta)}},
$$ 
gagne \`a \^etre p\'enalis\'ee  par $trace[KJ^{-1}]$ pour approcher la d\'eviance
calcul\'ee sur un \'echantillon de taille infini et  obtenir un crit\`ere de choix 
de mod\`ele. Ainsi, l'\'equation \ref{eq:complexite} est \`a la base des crit\`eres
de choix de mod\`ele :
\begin{itemize}
\item NIC (Network Information Criterion) $= D_{\cal F}(\hat{\theta}) + 2d^*$, \\
avec $d^*=trace[KJ^{-1}]$. On peut approximer les matrices $K$ et 
$J$ en remplacant esp\'erance par moyenne sur ${\cal F}$ et $\theta_0$ par
$\hat{\theta}$.
\item AIC (An Information Criterion)$ = D_{\cal F}(\hat{\theta}) + 2d$ ; \\
o\`u $d^*$ est remplac\'e par $d$ le nombre de param\`etre du mod\`ele. 
Notons que cette approximation est justifi\'ee lorsque la densit\'e vraie
$f()$ appartient \`a la famille $f_{\theta}$, car dans ce cas $K=J$ et
$d^*=trace[I]$.
\end{itemize}

Le mod\`ele choisit par ce type d'approche est bien \'evidemment celui 
qui donne la plus petite valeur du crit\`ere choisit. En terme de vraisemblance,
cela revient \`a choisir le mod\`ele qui maximise la vraisemblance p\'enalis\'ee
par un terme n\'egatif, li\'e au nombre de param\`etres du mod\`ele.
 
\subsection{S\'election par  validation crois\'ee}

Il est aussi possible de recourir \`a  une proc\'edure de validation
crois\'ee pour choisir le meilleur mod\`ele au vue d'un 
certain crit\`ere. Si le crit\`ere consid\'er\'e est la d\'eviance
alors cela revient \`a utiliser le crit\`ere NIC.

Supposons que l'on dispose d'un \'echantillon ${\cal F}=(\x_1,\cdots,\x_N)$ et
notons ${\cal F}_i$, l'\'echantillon ${\cal F}$ sans $\x_i$. La proc\'edure
de validation crois\'ee consiste \`a estimer  $\hat{\theta}_i$ par 
maximum de vraisemblance en utilisant ${\cal F_i}$ et \`a calculer le terme 
$$
D_i(\hat{\theta}_i)=2( \log f(\x_i) - \log f(\x_i ; \hat{\theta}_i) )
$$
La validation crois\'ee de l'\'ecart au mod\`ele s'exprime alors comme
la somme de tous les $D_i$ obtenus. Si l'on consid\`ere le 
developpement de taylor de cette somme \`a l'ordre 1 en $\hat{\theta}$, 
on trouve :
\begin{eqnarray*}
 \sum_{i=1}^{N} D_i(\hat{\theta}_i)  & = &\sum_{i=1}^{N} D_i(\hat{\theta}) +  \sum_{i=1}^{N} ( \hat{\theta}_i- \hat{\theta})^t D_i^{'}(\tilde{\theta}_i) \\
          & = & D_{\cal F}(\hat{\theta}) +   \sum_{i=1}^{N} ( \hat{\theta}_i- \hat{\theta})^t D_i^{'}(\tilde{\theta}_i)
\end{eqnarray*}
avec $\tilde{\theta}_i$ une combinaison convexe de $\hat{\theta}_i$ et 
$\hat{\theta}$. Consid\'erons ensuite le d\'eveloppement de taylor
du vecteur $D_i^{'}(\hat{\theta}_i)$ (qui est le transpos\'e du gradient de
$D_i(\hat{\theta}_i)$ par rapport \`a $\hat{\theta}_i$) : 
\begin{eqnarray*}
D_i^{'}(\hat{\theta}_i)  & = &  D_i^{'}(\hat{\theta}) +  ( \hat{\theta}_i- \hat{\theta})^t D_i^{''}(\overline{\theta}_i) \\
D_i^{'}(\hat{\theta}_i)  & = &   ( \hat{\theta}_i- \hat{\theta})^t D_i^{''}(\overline{\theta}_i) 
\end{eqnarray*}
avec $\overline{\theta}_i$ une combinaison convexe de $\hat{\theta}_i$ et 
$\hat{\theta}$. Si tous les estimateurs converge vers $\theta_0$, alors
\begin{eqnarray*}
\sum_{i=1}^{N} D_i(\hat{\theta}_i)  & \approx  & D_{\cal F}(\hat{\theta}) +   
\sum_{i=1}^{N} D_i^{'}(\theta_0)^t D_i^{''}(\theta_0)^{-1}
 D_i^{'}(\theta_0) \\
  & \approx  & D_{\cal F}(\hat{\theta}) + trace [KJ^{-1}].
\end{eqnarray*}

Ce qui montre que cette approche est asymptotiquement \'equivalente au choix de mod\`ele 
sur la base du crit\`ere NIC.

% Choix de mod\`le
 % Critere de choix
 % Leave one out procedure
