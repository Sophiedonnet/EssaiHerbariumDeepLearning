\documentstyle[12pt,french]{article}
\setlength{\topmargin}{-2cm}
\setlength{\oddsidemargin}{1cm}
\setlength{\leftmargin}{-5cm}
\setlength{\textwidth}{14.3cm}
\setlength{\textheight}{23.8cm}
\setlength{\parindent}{0cm}
\def\thepage{}

%\title{Reconnaissance statistique des formes}
%\author{}
%\date{}

\begin{document}
%\maketitle
A compter du 11 mars 1997, et pendant 8 semaines, Christophe Ambroise
animera un s\'eminaire intitul\'e : 
\begin{center}
\Large
Reconnaissance statistique des formes
\end{center}
Les s\'eances de 2 heures se d\'erouleront le mardi  \`a 14 heures dans
le grand amphi du site de Fontainebleau.

\vspace{2cm}

{\bf Pr\'esentation :}\\
La reconnaissance des formes (``pattern recognition'' en anglais) est aussi
connue sous les noms de classification, diagnostic, apprentissage \`a 
base d'exemples. Cet ensemble de 8 s\'eances de 2 heures chacune est une revue
des techniques les plus courantes du domaine et insiste plus 
particuli\`erement sur les m\'ethodes non supervis\'ees :

\begin{description}
\item[11 mars 97 :] introduction : exemples et th\'eorie statistique 
 de la d\'ecision ; 
\item[18 mars 97 :] analyse discriminante param\'etrique ;
\item[25 mars 97 :] analyse discriminante non param\'etrique ;
\item[1 avril 97 :] r\'eseaux de neurones \`a couches ;
\item[8 avril 97 :] introduction aux techniques de classification automatique ;
\item[15 avril :] approche probabiliste en classification automatique ;
\item[22 avril 97 :] champs de Markov et traitement d'image ;
\item[29 avril 97 :] approche markovienne en segmentation d'image.
\end{description}

\end{document}